\documentclass[11pt]{article}

    \usepackage[breakable]{tcolorbox}
    \usepackage{parskip} % Stop auto-indenting (to mimic markdown behaviour)
    
    \usepackage{iftex}
    \ifPDFTeX
    	\usepackage[T1]{fontenc}
    	\usepackage{mathpazo}
    \else
    	\usepackage{fontspec}
    \fi

    % Basic figure setup, for now with no caption control since it's done
    % automatically by Pandoc (which extracts ![](path) syntax from Markdown).
    \usepackage{graphicx}
    % Maintain compatibility with old templates. Remove in nbconvert 6.0
    \let\Oldincludegraphics\includegraphics
    % Ensure that by default, figures have no caption (until we provide a
    % proper Figure object with a Caption API and a way to capture that
    % in the conversion process - todo).
    \usepackage{caption}
    \DeclareCaptionFormat{nocaption}{}
    \captionsetup{format=nocaption,aboveskip=0pt,belowskip=0pt}

    \usepackage{float}
    \floatplacement{figure}{H} % forces figures to be placed at the correct location
    \usepackage{xcolor} % Allow colors to be defined
    \usepackage{enumerate} % Needed for markdown enumerations to work
    \usepackage{geometry} % Used to adjust the document margins
    \usepackage{amsmath} % Equations
    \usepackage{amssymb} % Equations
    \usepackage{textcomp} % defines textquotesingle
    % Hack from http://tex.stackexchange.com/a/47451/13684:
    \AtBeginDocument{%
        \def\PYZsq{\textquotesingle}% Upright quotes in Pygmentized code
    }
    \usepackage{upquote} % Upright quotes for verbatim code
    \usepackage{eurosym} % defines \euro
    \usepackage[mathletters]{ucs} % Extended unicode (utf-8) support
    \usepackage{fancyvrb} % verbatim replacement that allows latex
    \usepackage{grffile} % extends the file name processing of package graphics 
                         % to support a larger range
    \makeatletter % fix for old versions of grffile with XeLaTeX
    \@ifpackagelater{grffile}{2019/11/01}
    {
      % Do nothing on new versions
    }
    {
      \def\Gread@@xetex#1{%
        \IfFileExists{"\Gin@base".bb}%
        {\Gread@eps{\Gin@base.bb}}%
        {\Gread@@xetex@aux#1}%
      }
    }
    \makeatother
    \usepackage[Export]{adjustbox} % Used to constrain images to a maximum size
    \adjustboxset{max size={0.9\linewidth}{0.9\paperheight}}

    % The hyperref package gives us a pdf with properly built
    % internal navigation ('pdf bookmarks' for the table of contents,
    % internal cross-reference links, web links for URLs, etc.)
    \usepackage{hyperref}
    % The default LaTeX title has an obnoxious amount of whitespace. By default,
    % titling removes some of it. It also provides customization options.
    \usepackage{titling}
    \usepackage{longtable} % longtable support required by pandoc >1.10
    \usepackage{booktabs}  % table support for pandoc > 1.12.2
    \usepackage[inline]{enumitem} % IRkernel/repr support (it uses the enumerate* environment)
    \usepackage[normalem]{ulem} % ulem is needed to support strikethroughs (\sout)
                                % normalem makes italics be italics, not underlines
    \usepackage{mathrsfs}
    

    
    % Colors for the hyperref package
    \definecolor{urlcolor}{rgb}{0,.145,.698}
    \definecolor{linkcolor}{rgb}{.71,0.21,0.01}
    \definecolor{citecolor}{rgb}{.12,.54,.11}

    % ANSI colors
    \definecolor{ansi-black}{HTML}{3E424D}
    \definecolor{ansi-black-intense}{HTML}{282C36}
    \definecolor{ansi-red}{HTML}{E75C58}
    \definecolor{ansi-red-intense}{HTML}{B22B31}
    \definecolor{ansi-green}{HTML}{00A250}
    \definecolor{ansi-green-intense}{HTML}{007427}
    \definecolor{ansi-yellow}{HTML}{DDB62B}
    \definecolor{ansi-yellow-intense}{HTML}{B27D12}
    \definecolor{ansi-blue}{HTML}{208FFB}
    \definecolor{ansi-blue-intense}{HTML}{0065CA}
    \definecolor{ansi-magenta}{HTML}{D160C4}
    \definecolor{ansi-magenta-intense}{HTML}{A03196}
    \definecolor{ansi-cyan}{HTML}{60C6C8}
    \definecolor{ansi-cyan-intense}{HTML}{258F8F}
    \definecolor{ansi-white}{HTML}{C5C1B4}
    \definecolor{ansi-white-intense}{HTML}{A1A6B2}
    \definecolor{ansi-default-inverse-fg}{HTML}{FFFFFF}
    \definecolor{ansi-default-inverse-bg}{HTML}{000000}

    % common color for the border for error outputs.
    \definecolor{outerrorbackground}{HTML}{FFDFDF}

    % commands and environments needed by pandoc snippets
    % extracted from the output of `pandoc -s`
    \providecommand{\tightlist}{%
      \setlength{\itemsep}{0pt}\setlength{\parskip}{0pt}}
    \DefineVerbatimEnvironment{Highlighting}{Verbatim}{commandchars=\\\{\}}
    % Add ',fontsize=\small' for more characters per line
    \newenvironment{Shaded}{}{}
    \newcommand{\KeywordTok}[1]{\textcolor[rgb]{0.00,0.44,0.13}{\textbf{{#1}}}}
    \newcommand{\DataTypeTok}[1]{\textcolor[rgb]{0.56,0.13,0.00}{{#1}}}
    \newcommand{\DecValTok}[1]{\textcolor[rgb]{0.25,0.63,0.44}{{#1}}}
    \newcommand{\BaseNTok}[1]{\textcolor[rgb]{0.25,0.63,0.44}{{#1}}}
    \newcommand{\FloatTok}[1]{\textcolor[rgb]{0.25,0.63,0.44}{{#1}}}
    \newcommand{\CharTok}[1]{\textcolor[rgb]{0.25,0.44,0.63}{{#1}}}
    \newcommand{\StringTok}[1]{\textcolor[rgb]{0.25,0.44,0.63}{{#1}}}
    \newcommand{\CommentTok}[1]{\textcolor[rgb]{0.38,0.63,0.69}{\textit{{#1}}}}
    \newcommand{\OtherTok}[1]{\textcolor[rgb]{0.00,0.44,0.13}{{#1}}}
    \newcommand{\AlertTok}[1]{\textcolor[rgb]{1.00,0.00,0.00}{\textbf{{#1}}}}
    \newcommand{\FunctionTok}[1]{\textcolor[rgb]{0.02,0.16,0.49}{{#1}}}
    \newcommand{\RegionMarkerTok}[1]{{#1}}
    \newcommand{\ErrorTok}[1]{\textcolor[rgb]{1.00,0.00,0.00}{\textbf{{#1}}}}
    \newcommand{\NormalTok}[1]{{#1}}
    
    % Additional commands for more recent versions of Pandoc
    \newcommand{\ConstantTok}[1]{\textcolor[rgb]{0.53,0.00,0.00}{{#1}}}
    \newcommand{\SpecialCharTok}[1]{\textcolor[rgb]{0.25,0.44,0.63}{{#1}}}
    \newcommand{\VerbatimStringTok}[1]{\textcolor[rgb]{0.25,0.44,0.63}{{#1}}}
    \newcommand{\SpecialStringTok}[1]{\textcolor[rgb]{0.73,0.40,0.53}{{#1}}}
    \newcommand{\ImportTok}[1]{{#1}}
    \newcommand{\DocumentationTok}[1]{\textcolor[rgb]{0.73,0.13,0.13}{\textit{{#1}}}}
    \newcommand{\AnnotationTok}[1]{\textcolor[rgb]{0.38,0.63,0.69}{\textbf{\textit{{#1}}}}}
    \newcommand{\CommentVarTok}[1]{\textcolor[rgb]{0.38,0.63,0.69}{\textbf{\textit{{#1}}}}}
    \newcommand{\VariableTok}[1]{\textcolor[rgb]{0.10,0.09,0.49}{{#1}}}
    \newcommand{\ControlFlowTok}[1]{\textcolor[rgb]{0.00,0.44,0.13}{\textbf{{#1}}}}
    \newcommand{\OperatorTok}[1]{\textcolor[rgb]{0.40,0.40,0.40}{{#1}}}
    \newcommand{\BuiltInTok}[1]{{#1}}
    \newcommand{\ExtensionTok}[1]{{#1}}
    \newcommand{\PreprocessorTok}[1]{\textcolor[rgb]{0.74,0.48,0.00}{{#1}}}
    \newcommand{\AttributeTok}[1]{\textcolor[rgb]{0.49,0.56,0.16}{{#1}}}
    \newcommand{\InformationTok}[1]{\textcolor[rgb]{0.38,0.63,0.69}{\textbf{\textit{{#1}}}}}
    \newcommand{\WarningTok}[1]{\textcolor[rgb]{0.38,0.63,0.69}{\textbf{\textit{{#1}}}}}
    
    
    % Define a nice break command that doesn't care if a line doesn't already
    % exist.
    \def\br{\hspace*{\fill} \\* }
    % Math Jax compatibility definitions
    \def\gt{>}
    \def\lt{<}
    \let\Oldtex\TeX
    \let\Oldlatex\LaTeX
    \renewcommand{\TeX}{\textrm{\Oldtex}}
    \renewcommand{\LaTeX}{\textrm{\Oldlatex}}
    % Document parameters
    % Document title
    \title{R\_APPENDIX-B\_MicroeconometricsR}
    
    
    
    
    
% Pygments definitions
\makeatletter
\def\PY@reset{\let\PY@it=\relax \let\PY@bf=\relax%
    \let\PY@ul=\relax \let\PY@tc=\relax%
    \let\PY@bc=\relax \let\PY@ff=\relax}
\def\PY@tok#1{\csname PY@tok@#1\endcsname}
\def\PY@toks#1+{\ifx\relax#1\empty\else%
    \PY@tok{#1}\expandafter\PY@toks\fi}
\def\PY@do#1{\PY@bc{\PY@tc{\PY@ul{%
    \PY@it{\PY@bf{\PY@ff{#1}}}}}}}
\def\PY#1#2{\PY@reset\PY@toks#1+\relax+\PY@do{#2}}

\@namedef{PY@tok@w}{\def\PY@tc##1{\textcolor[rgb]{0.73,0.73,0.73}{##1}}}
\@namedef{PY@tok@c}{\let\PY@it=\textit\def\PY@tc##1{\textcolor[rgb]{0.25,0.50,0.50}{##1}}}
\@namedef{PY@tok@cp}{\def\PY@tc##1{\textcolor[rgb]{0.74,0.48,0.00}{##1}}}
\@namedef{PY@tok@k}{\let\PY@bf=\textbf\def\PY@tc##1{\textcolor[rgb]{0.00,0.50,0.00}{##1}}}
\@namedef{PY@tok@kp}{\def\PY@tc##1{\textcolor[rgb]{0.00,0.50,0.00}{##1}}}
\@namedef{PY@tok@kt}{\def\PY@tc##1{\textcolor[rgb]{0.69,0.00,0.25}{##1}}}
\@namedef{PY@tok@o}{\def\PY@tc##1{\textcolor[rgb]{0.40,0.40,0.40}{##1}}}
\@namedef{PY@tok@ow}{\let\PY@bf=\textbf\def\PY@tc##1{\textcolor[rgb]{0.67,0.13,1.00}{##1}}}
\@namedef{PY@tok@nb}{\def\PY@tc##1{\textcolor[rgb]{0.00,0.50,0.00}{##1}}}
\@namedef{PY@tok@nf}{\def\PY@tc##1{\textcolor[rgb]{0.00,0.00,1.00}{##1}}}
\@namedef{PY@tok@nc}{\let\PY@bf=\textbf\def\PY@tc##1{\textcolor[rgb]{0.00,0.00,1.00}{##1}}}
\@namedef{PY@tok@nn}{\let\PY@bf=\textbf\def\PY@tc##1{\textcolor[rgb]{0.00,0.00,1.00}{##1}}}
\@namedef{PY@tok@ne}{\let\PY@bf=\textbf\def\PY@tc##1{\textcolor[rgb]{0.82,0.25,0.23}{##1}}}
\@namedef{PY@tok@nv}{\def\PY@tc##1{\textcolor[rgb]{0.10,0.09,0.49}{##1}}}
\@namedef{PY@tok@no}{\def\PY@tc##1{\textcolor[rgb]{0.53,0.00,0.00}{##1}}}
\@namedef{PY@tok@nl}{\def\PY@tc##1{\textcolor[rgb]{0.63,0.63,0.00}{##1}}}
\@namedef{PY@tok@ni}{\let\PY@bf=\textbf\def\PY@tc##1{\textcolor[rgb]{0.60,0.60,0.60}{##1}}}
\@namedef{PY@tok@na}{\def\PY@tc##1{\textcolor[rgb]{0.49,0.56,0.16}{##1}}}
\@namedef{PY@tok@nt}{\let\PY@bf=\textbf\def\PY@tc##1{\textcolor[rgb]{0.00,0.50,0.00}{##1}}}
\@namedef{PY@tok@nd}{\def\PY@tc##1{\textcolor[rgb]{0.67,0.13,1.00}{##1}}}
\@namedef{PY@tok@s}{\def\PY@tc##1{\textcolor[rgb]{0.73,0.13,0.13}{##1}}}
\@namedef{PY@tok@sd}{\let\PY@it=\textit\def\PY@tc##1{\textcolor[rgb]{0.73,0.13,0.13}{##1}}}
\@namedef{PY@tok@si}{\let\PY@bf=\textbf\def\PY@tc##1{\textcolor[rgb]{0.73,0.40,0.53}{##1}}}
\@namedef{PY@tok@se}{\let\PY@bf=\textbf\def\PY@tc##1{\textcolor[rgb]{0.73,0.40,0.13}{##1}}}
\@namedef{PY@tok@sr}{\def\PY@tc##1{\textcolor[rgb]{0.73,0.40,0.53}{##1}}}
\@namedef{PY@tok@ss}{\def\PY@tc##1{\textcolor[rgb]{0.10,0.09,0.49}{##1}}}
\@namedef{PY@tok@sx}{\def\PY@tc##1{\textcolor[rgb]{0.00,0.50,0.00}{##1}}}
\@namedef{PY@tok@m}{\def\PY@tc##1{\textcolor[rgb]{0.40,0.40,0.40}{##1}}}
\@namedef{PY@tok@gh}{\let\PY@bf=\textbf\def\PY@tc##1{\textcolor[rgb]{0.00,0.00,0.50}{##1}}}
\@namedef{PY@tok@gu}{\let\PY@bf=\textbf\def\PY@tc##1{\textcolor[rgb]{0.50,0.00,0.50}{##1}}}
\@namedef{PY@tok@gd}{\def\PY@tc##1{\textcolor[rgb]{0.63,0.00,0.00}{##1}}}
\@namedef{PY@tok@gi}{\def\PY@tc##1{\textcolor[rgb]{0.00,0.63,0.00}{##1}}}
\@namedef{PY@tok@gr}{\def\PY@tc##1{\textcolor[rgb]{1.00,0.00,0.00}{##1}}}
\@namedef{PY@tok@ge}{\let\PY@it=\textit}
\@namedef{PY@tok@gs}{\let\PY@bf=\textbf}
\@namedef{PY@tok@gp}{\let\PY@bf=\textbf\def\PY@tc##1{\textcolor[rgb]{0.00,0.00,0.50}{##1}}}
\@namedef{PY@tok@go}{\def\PY@tc##1{\textcolor[rgb]{0.53,0.53,0.53}{##1}}}
\@namedef{PY@tok@gt}{\def\PY@tc##1{\textcolor[rgb]{0.00,0.27,0.87}{##1}}}
\@namedef{PY@tok@err}{\def\PY@bc##1{{\setlength{\fboxsep}{\string -\fboxrule}\fcolorbox[rgb]{1.00,0.00,0.00}{1,1,1}{\strut ##1}}}}
\@namedef{PY@tok@kc}{\let\PY@bf=\textbf\def\PY@tc##1{\textcolor[rgb]{0.00,0.50,0.00}{##1}}}
\@namedef{PY@tok@kd}{\let\PY@bf=\textbf\def\PY@tc##1{\textcolor[rgb]{0.00,0.50,0.00}{##1}}}
\@namedef{PY@tok@kn}{\let\PY@bf=\textbf\def\PY@tc##1{\textcolor[rgb]{0.00,0.50,0.00}{##1}}}
\@namedef{PY@tok@kr}{\let\PY@bf=\textbf\def\PY@tc##1{\textcolor[rgb]{0.00,0.50,0.00}{##1}}}
\@namedef{PY@tok@bp}{\def\PY@tc##1{\textcolor[rgb]{0.00,0.50,0.00}{##1}}}
\@namedef{PY@tok@fm}{\def\PY@tc##1{\textcolor[rgb]{0.00,0.00,1.00}{##1}}}
\@namedef{PY@tok@vc}{\def\PY@tc##1{\textcolor[rgb]{0.10,0.09,0.49}{##1}}}
\@namedef{PY@tok@vg}{\def\PY@tc##1{\textcolor[rgb]{0.10,0.09,0.49}{##1}}}
\@namedef{PY@tok@vi}{\def\PY@tc##1{\textcolor[rgb]{0.10,0.09,0.49}{##1}}}
\@namedef{PY@tok@vm}{\def\PY@tc##1{\textcolor[rgb]{0.10,0.09,0.49}{##1}}}
\@namedef{PY@tok@sa}{\def\PY@tc##1{\textcolor[rgb]{0.73,0.13,0.13}{##1}}}
\@namedef{PY@tok@sb}{\def\PY@tc##1{\textcolor[rgb]{0.73,0.13,0.13}{##1}}}
\@namedef{PY@tok@sc}{\def\PY@tc##1{\textcolor[rgb]{0.73,0.13,0.13}{##1}}}
\@namedef{PY@tok@dl}{\def\PY@tc##1{\textcolor[rgb]{0.73,0.13,0.13}{##1}}}
\@namedef{PY@tok@s2}{\def\PY@tc##1{\textcolor[rgb]{0.73,0.13,0.13}{##1}}}
\@namedef{PY@tok@sh}{\def\PY@tc##1{\textcolor[rgb]{0.73,0.13,0.13}{##1}}}
\@namedef{PY@tok@s1}{\def\PY@tc##1{\textcolor[rgb]{0.73,0.13,0.13}{##1}}}
\@namedef{PY@tok@mb}{\def\PY@tc##1{\textcolor[rgb]{0.40,0.40,0.40}{##1}}}
\@namedef{PY@tok@mf}{\def\PY@tc##1{\textcolor[rgb]{0.40,0.40,0.40}{##1}}}
\@namedef{PY@tok@mh}{\def\PY@tc##1{\textcolor[rgb]{0.40,0.40,0.40}{##1}}}
\@namedef{PY@tok@mi}{\def\PY@tc##1{\textcolor[rgb]{0.40,0.40,0.40}{##1}}}
\@namedef{PY@tok@il}{\def\PY@tc##1{\textcolor[rgb]{0.40,0.40,0.40}{##1}}}
\@namedef{PY@tok@mo}{\def\PY@tc##1{\textcolor[rgb]{0.40,0.40,0.40}{##1}}}
\@namedef{PY@tok@ch}{\let\PY@it=\textit\def\PY@tc##1{\textcolor[rgb]{0.25,0.50,0.50}{##1}}}
\@namedef{PY@tok@cm}{\let\PY@it=\textit\def\PY@tc##1{\textcolor[rgb]{0.25,0.50,0.50}{##1}}}
\@namedef{PY@tok@cpf}{\let\PY@it=\textit\def\PY@tc##1{\textcolor[rgb]{0.25,0.50,0.50}{##1}}}
\@namedef{PY@tok@c1}{\let\PY@it=\textit\def\PY@tc##1{\textcolor[rgb]{0.25,0.50,0.50}{##1}}}
\@namedef{PY@tok@cs}{\let\PY@it=\textit\def\PY@tc##1{\textcolor[rgb]{0.25,0.50,0.50}{##1}}}

\def\PYZbs{\char`\\}
\def\PYZus{\char`\_}
\def\PYZob{\char`\{}
\def\PYZcb{\char`\}}
\def\PYZca{\char`\^}
\def\PYZam{\char`\&}
\def\PYZlt{\char`\<}
\def\PYZgt{\char`\>}
\def\PYZsh{\char`\#}
\def\PYZpc{\char`\%}
\def\PYZdl{\char`\$}
\def\PYZhy{\char`\-}
\def\PYZsq{\char`\'}
\def\PYZdq{\char`\"}
\def\PYZti{\char`\~}
% for compatibility with earlier versions
\def\PYZat{@}
\def\PYZlb{[}
\def\PYZrb{]}
\makeatother


    % For linebreaks inside Verbatim environment from package fancyvrb. 
    \makeatletter
        \newbox\Wrappedcontinuationbox 
        \newbox\Wrappedvisiblespacebox 
        \newcommand*\Wrappedvisiblespace {\textcolor{red}{\textvisiblespace}} 
        \newcommand*\Wrappedcontinuationsymbol {\textcolor{red}{\llap{\tiny$\m@th\hookrightarrow$}}} 
        \newcommand*\Wrappedcontinuationindent {3ex } 
        \newcommand*\Wrappedafterbreak {\kern\Wrappedcontinuationindent\copy\Wrappedcontinuationbox} 
        % Take advantage of the already applied Pygments mark-up to insert 
        % potential linebreaks for TeX processing. 
        %        {, <, #, %, $, ' and ": go to next line. 
        %        _, }, ^, &, >, - and ~: stay at end of broken line. 
        % Use of \textquotesingle for straight quote. 
        \newcommand*\Wrappedbreaksatspecials {% 
            \def\PYGZus{\discretionary{\char`\_}{\Wrappedafterbreak}{\char`\_}}% 
            \def\PYGZob{\discretionary{}{\Wrappedafterbreak\char`\{}{\char`\{}}% 
            \def\PYGZcb{\discretionary{\char`\}}{\Wrappedafterbreak}{\char`\}}}% 
            \def\PYGZca{\discretionary{\char`\^}{\Wrappedafterbreak}{\char`\^}}% 
            \def\PYGZam{\discretionary{\char`\&}{\Wrappedafterbreak}{\char`\&}}% 
            \def\PYGZlt{\discretionary{}{\Wrappedafterbreak\char`\<}{\char`\<}}% 
            \def\PYGZgt{\discretionary{\char`\>}{\Wrappedafterbreak}{\char`\>}}% 
            \def\PYGZsh{\discretionary{}{\Wrappedafterbreak\char`\#}{\char`\#}}% 
            \def\PYGZpc{\discretionary{}{\Wrappedafterbreak\char`\%}{\char`\%}}% 
            \def\PYGZdl{\discretionary{}{\Wrappedafterbreak\char`\$}{\char`\$}}% 
            \def\PYGZhy{\discretionary{\char`\-}{\Wrappedafterbreak}{\char`\-}}% 
            \def\PYGZsq{\discretionary{}{\Wrappedafterbreak\textquotesingle}{\textquotesingle}}% 
            \def\PYGZdq{\discretionary{}{\Wrappedafterbreak\char`\"}{\char`\"}}% 
            \def\PYGZti{\discretionary{\char`\~}{\Wrappedafterbreak}{\char`\~}}% 
        } 
        % Some characters . , ; ? ! / are not pygmentized. 
        % This macro makes them "active" and they will insert potential linebreaks 
        \newcommand*\Wrappedbreaksatpunct {% 
            \lccode`\~`\.\lowercase{\def~}{\discretionary{\hbox{\char`\.}}{\Wrappedafterbreak}{\hbox{\char`\.}}}% 
            \lccode`\~`\,\lowercase{\def~}{\discretionary{\hbox{\char`\,}}{\Wrappedafterbreak}{\hbox{\char`\,}}}% 
            \lccode`\~`\;\lowercase{\def~}{\discretionary{\hbox{\char`\;}}{\Wrappedafterbreak}{\hbox{\char`\;}}}% 
            \lccode`\~`\:\lowercase{\def~}{\discretionary{\hbox{\char`\:}}{\Wrappedafterbreak}{\hbox{\char`\:}}}% 
            \lccode`\~`\?\lowercase{\def~}{\discretionary{\hbox{\char`\?}}{\Wrappedafterbreak}{\hbox{\char`\?}}}% 
            \lccode`\~`\!\lowercase{\def~}{\discretionary{\hbox{\char`\!}}{\Wrappedafterbreak}{\hbox{\char`\!}}}% 
            \lccode`\~`\/\lowercase{\def~}{\discretionary{\hbox{\char`\/}}{\Wrappedafterbreak}{\hbox{\char`\/}}}% 
            \catcode`\.\active
            \catcode`\,\active 
            \catcode`\;\active
            \catcode`\:\active
            \catcode`\?\active
            \catcode`\!\active
            \catcode`\/\active 
            \lccode`\~`\~ 	
        }
    \makeatother

    \let\OriginalVerbatim=\Verbatim
    \makeatletter
    \renewcommand{\Verbatim}[1][1]{%
        %\parskip\z@skip
        \sbox\Wrappedcontinuationbox {\Wrappedcontinuationsymbol}%
        \sbox\Wrappedvisiblespacebox {\FV@SetupFont\Wrappedvisiblespace}%
        \def\FancyVerbFormatLine ##1{\hsize\linewidth
            \vtop{\raggedright\hyphenpenalty\z@\exhyphenpenalty\z@
                \doublehyphendemerits\z@\finalhyphendemerits\z@
                \strut ##1\strut}%
        }%
        % If the linebreak is at a space, the latter will be displayed as visible
        % space at end of first line, and a continuation symbol starts next line.
        % Stretch/shrink are however usually zero for typewriter font.
        \def\FV@Space {%
            \nobreak\hskip\z@ plus\fontdimen3\font minus\fontdimen4\font
            \discretionary{\copy\Wrappedvisiblespacebox}{\Wrappedafterbreak}
            {\kern\fontdimen2\font}%
        }%
        
        % Allow breaks at special characters using \PYG... macros.
        \Wrappedbreaksatspecials
        % Breaks at punctuation characters . , ; ? ! and / need catcode=\active 	
        \OriginalVerbatim[#1,codes*=\Wrappedbreaksatpunct]%
    }
    \makeatother

    % Exact colors from NB
    \definecolor{incolor}{HTML}{303F9F}
    \definecolor{outcolor}{HTML}{D84315}
    \definecolor{cellborder}{HTML}{CFCFCF}
    \definecolor{cellbackground}{HTML}{F7F7F7}
    
    % prompt
    \makeatletter
    \newcommand{\boxspacing}{\kern\kvtcb@left@rule\kern\kvtcb@boxsep}
    \makeatother
    \newcommand{\prompt}[4]{
        {\ttfamily\llap{{\color{#2}[#3]:\hspace{3pt}#4}}\vspace{-\baselineskip}}
    }
    

    
    % Prevent overflowing lines due to hard-to-break entities
    \sloppy 
    % Setup hyperref package
    \hypersetup{
      breaklinks=true,  % so long urls are correctly broken across lines
      colorlinks=true,
      urlcolor=urlcolor,
      linkcolor=linkcolor,
      citecolor=citecolor,
      }
    % Slightly bigger margins than the latex defaults
    
    \geometry{verbose,tmargin=1in,bmargin=1in,lmargin=1in,rmargin=1in}
    
    

\begin{document}
    
    \maketitle
    
    

    
    \begin{tcolorbox}[breakable, size=fbox, boxrule=1pt, pad at break*=1mm,colback=cellbackground, colframe=cellborder]
\prompt{In}{incolor}{2}{\boxspacing}
\begin{Verbatim}[commandchars=\\\{\}]
\PY{c+c1}{\PYZsh{} B2. Objects in R}

\PY{c+c1}{\PYZsh{} B2.1. Vectors}

\PY{n}{a} \PY{o}{\PYZlt{}\PYZhy{}} \PY{n+nf}{c}\PY{p}{(}\PY{l+m}{1}\PY{p}{,}\PY{l+m}{2}\PY{p}{,}\PY{l+m}{3}\PY{p}{)}
\PY{n}{b} \PY{o}{\PYZlt{}\PYZhy{}} \PY{n+nf}{c}\PY{p}{(}\PY{l+s}{\PYZdq{}}\PY{l+s}{one\PYZdq{}}\PY{p}{,} \PY{l+s}{\PYZdq{}}\PY{l+s}{two\PYZdq{}}\PY{p}{,} \PY{l+s}{\PYZdq{}}\PY{l+s}{three\PYZdq{}}\PY{p}{)}

\PY{n}{a}

\PY{n}{b}
\end{Verbatim}
\end{tcolorbox}

    \begin{enumerate*}
\item 1
\item 2
\item 3
\end{enumerate*}


    
    \begin{enumerate*}
\item 'one'
\item 'two'
\item 'three'
\end{enumerate*}


    
    \begin{tcolorbox}[breakable, size=fbox, boxrule=1pt, pad at break*=1mm,colback=cellbackground, colframe=cellborder]
\prompt{In}{incolor}{4}{\boxspacing}
\begin{Verbatim}[commandchars=\\\{\}]
\PY{n}{d} \PY{o}{\PYZlt{}\PYZhy{}} \PY{n+nf}{c}\PY{p}{(}\PY{n}{a}\PY{p}{,}\PY{n}{b}\PY{p}{)}

\PY{n}{d}
\end{Verbatim}
\end{tcolorbox}

    \begin{enumerate*}
\item '1'
\item '2'
\item '3'
\item 'one'
\item 'two'
\item 'three'
\end{enumerate*}


    
    \begin{tcolorbox}[breakable, size=fbox, boxrule=1pt, pad at break*=1mm,colback=cellbackground, colframe=cellborder]
\prompt{In}{incolor}{5}{\boxspacing}
\begin{Verbatim}[commandchars=\\\{\}]
\PY{n}{b} \PY{o}{\PYZlt{}\PYZhy{}} \PY{n+nf}{c}\PY{p}{(}\PY{l+m}{4}\PY{p}{,}\PY{l+m}{5}\PY{p}{,}\PY{l+m}{6}\PY{p}{)}
\PY{n}{a}\PY{o}{+}\PY{n}{b}
\end{Verbatim}
\end{tcolorbox}

    \begin{enumerate*}
\item 5
\item 7
\item 9
\end{enumerate*}


    
    \begin{tcolorbox}[breakable, size=fbox, boxrule=1pt, pad at break*=1mm,colback=cellbackground, colframe=cellborder]
\prompt{In}{incolor}{6}{\boxspacing}
\begin{Verbatim}[commandchars=\\\{\}]
\PY{n}{a}\PY{o}{*}\PY{n}{b}
\end{Verbatim}
\end{tcolorbox}

    \begin{enumerate*}
\item 4
\item 10
\item 18
\end{enumerate*}


    
    \begin{tcolorbox}[breakable, size=fbox, boxrule=1pt, pad at break*=1mm,colback=cellbackground, colframe=cellborder]
\prompt{In}{incolor}{7}{\boxspacing}
\begin{Verbatim}[commandchars=\\\{\}]
\PY{n}{a}\PY{o}{/}\PY{n}{b}
\end{Verbatim}
\end{tcolorbox}

    \begin{enumerate*}
\item 0.25
\item 0.4
\item 0.5
\end{enumerate*}


    
    \begin{tcolorbox}[breakable, size=fbox, boxrule=1pt, pad at break*=1mm,colback=cellbackground, colframe=cellborder]
\prompt{In}{incolor}{8}{\boxspacing}
\begin{Verbatim}[commandchars=\\\{\}]
\PY{n}{a} \PY{o}{+} \PY{l+m}{2}
\end{Verbatim}
\end{tcolorbox}

    \begin{enumerate*}
\item 3
\item 4
\item 5
\end{enumerate*}


    
    \begin{tcolorbox}[breakable, size=fbox, boxrule=1pt, pad at break*=1mm,colback=cellbackground, colframe=cellborder]
\prompt{In}{incolor}{9}{\boxspacing}
\begin{Verbatim}[commandchars=\\\{\}]
\PY{n}{a}\PY{o}{*}\PY{l+m}{2}
\end{Verbatim}
\end{tcolorbox}

    \begin{enumerate*}
\item 2
\item 4
\item 6
\end{enumerate*}


    
    \begin{tcolorbox}[breakable, size=fbox, boxrule=1pt, pad at break*=1mm,colback=cellbackground, colframe=cellborder]
\prompt{In}{incolor}{10}{\boxspacing}
\begin{Verbatim}[commandchars=\\\{\}]
\PY{n}{a}\PY{o}{/}\PY{l+m}{2}
\end{Verbatim}
\end{tcolorbox}

    \begin{enumerate*}
\item 0.5
\item 1
\item 1.5
\end{enumerate*}


    
    \begin{tcolorbox}[breakable, size=fbox, boxrule=1pt, pad at break*=1mm,colback=cellbackground, colframe=cellborder]
\prompt{In}{incolor}{11}{\boxspacing}
\begin{Verbatim}[commandchars=\\\{\}]
\PY{n}{b} \PY{o}{\PYZlt{}\PYZhy{}} \PY{n+nf}{c}\PY{p}{(}\PY{l+m}{4}\PY{p}{,}\PY{l+m}{5}\PY{p}{)}
\PY{n}{a}\PY{o}{+}\PY{n}{b}

\PY{c+c1}{\PYZsh{} Important warning that R still do the operation regardless length!}
\end{Verbatim}
\end{tcolorbox}

    \begin{Verbatim}[commandchars=\\\{\}]
Warning message in a + b:
"longer object length is not a multiple of shorter object length"
    \end{Verbatim}

    \begin{enumerate*}
\item 5
\item 7
\item 7
\end{enumerate*}


    
    \begin{tcolorbox}[breakable, size=fbox, boxrule=1pt, pad at break*=1mm,colback=cellbackground, colframe=cellborder]
\prompt{In}{incolor}{12}{\boxspacing}
\begin{Verbatim}[commandchars=\\\{\}]
\PY{n}{a}\PY{o}{*}\PY{n}{b}

\PY{c+c1}{\PYZsh{} Important warning that R still do the operation regardless length!}
\end{Verbatim}
\end{tcolorbox}

    \begin{Verbatim}[commandchars=\\\{\}]
Warning message in a * b:
"longer object length is not a multiple of shorter object length"
    \end{Verbatim}

    \begin{enumerate*}
\item 4
\item 10
\item 12
\end{enumerate*}


    
    \begin{tcolorbox}[breakable, size=fbox, boxrule=1pt, pad at break*=1mm,colback=cellbackground, colframe=cellborder]
\prompt{In}{incolor}{14}{\boxspacing}
\begin{Verbatim}[commandchars=\\\{\}]
\PY{c+c1}{\PYZsh{} B2.2. Matrices}

\PY{n}{a} \PY{o}{\PYZlt{}\PYZhy{}} \PY{n+nf}{c}\PY{p}{(}\PY{l+m}{1}\PY{p}{,}\PY{l+m}{2}\PY{p}{,}\PY{l+m}{3}\PY{p}{)}
\PY{n}{b} \PY{o}{\PYZlt{}\PYZhy{}} \PY{n+nf}{c}\PY{p}{(}\PY{l+m}{4}\PY{p}{,}\PY{l+m}{5}\PY{p}{,}\PY{l+m}{6}\PY{p}{)}
\PY{n}{A} \PY{o}{\PYZlt{}\PYZhy{}} \PY{n+nf}{cbind}\PY{p}{(}\PY{n}{a}\PY{p}{,}\PY{n}{b}\PY{p}{)}
\PY{n}{B} \PY{o}{\PYZlt{}\PYZhy{}} \PY{n+nf}{rbind}\PY{p}{(}\PY{n}{a}\PY{p}{,}\PY{n}{b}\PY{p}{)}

\PY{n}{A}

\PY{n+nf}{is.matrix}\PY{p}{(}\PY{n}{A}\PY{p}{)}

\PY{n}{B}

\PY{n+nf}{is.matrix}\PY{p}{(}\PY{n}{B}\PY{p}{)}
\end{Verbatim}
\end{tcolorbox}

    \begin{tabular}{ll}
 a & b\\
\hline
	 1 & 4\\
	 2 & 5\\
	 3 & 6\\
\end{tabular}


    
    TRUE

    
    \begin{tabular}{r|lll}
	a & 1 & 2 & 3\\
	b & 4 & 5 & 6\\
\end{tabular}


    
    TRUE

    
    \begin{tcolorbox}[breakable, size=fbox, boxrule=1pt, pad at break*=1mm,colback=cellbackground, colframe=cellborder]
\prompt{In}{incolor}{23}{\boxspacing}
\begin{Verbatim}[commandchars=\\\{\}]
\PY{n+nf}{t}\PY{p}{(}\PY{n}{A}\PY{p}{)}

\PY{n+nf}{t}\PY{p}{(}\PY{n}{B}\PY{p}{)}
\end{Verbatim}
\end{tcolorbox}

    \begin{tabular}{r|lll}
	a & 1 & 2 & 3\\
	b & 4 & 5 & 6\\
\end{tabular}


    
    \begin{tabular}{ll}
 a & b\\
\hline
	 1 & 4\\
	 2 & 5\\
	 3 & 6\\
\end{tabular}


    
    \begin{tcolorbox}[breakable, size=fbox, boxrule=1pt, pad at break*=1mm,colback=cellbackground, colframe=cellborder]
\prompt{In}{incolor}{19}{\boxspacing}
\begin{Verbatim}[commandchars=\\\{\}]
\PY{n}{C} \PY{o}{\PYZlt{}\PYZhy{}} \PY{n}{A} \PY{o}{+} \PY{l+m}{2}
\PY{n}{C}
\PY{n}{A} \PY{o}{+} \PY{n}{C}  \PY{c+c1}{\PYZsh{} cell\PYZhy{}by\PYZhy{}cell operations}
\end{Verbatim}
\end{tcolorbox}

    \begin{tabular}{ll}
 a & b\\
\hline
	 3 & 6\\
	 4 & 7\\
	 5 & 8\\
\end{tabular}


    
    \begin{tabular}{ll}
 a & b\\
\hline
	 4  & 10\\
	 6  & 12\\
	 8  & 14\\
\end{tabular}


    
    \begin{tcolorbox}[breakable, size=fbox, boxrule=1pt, pad at break*=1mm,colback=cellbackground, colframe=cellborder]
\prompt{In}{incolor}{20}{\boxspacing}
\begin{Verbatim}[commandchars=\\\{\}]
\PY{n}{D} \PY{o}{\PYZlt{}\PYZhy{}} \PY{n}{B}\PY{o}{*}\PY{l+m}{2}
\PY{n}{D}
\PY{n}{B}\PY{o}{*}\PY{n}{D}  \PY{c+c1}{\PYZsh{} cell\PYZhy{}by\PYZhy{}cell operations}
\end{Verbatim}
\end{tcolorbox}

    \begin{tabular}{r|lll}
	a & 2  &  4 &  6\\
	b & 8  & 10 & 12\\
\end{tabular}


    
    \begin{tabular}{r|lll}
	a &  2 &  8 & 18\\
	b & 32 & 50 & 72\\
\end{tabular}


    
    \begin{tcolorbox}[breakable, size=fbox, boxrule=1pt, pad at break*=1mm,colback=cellbackground, colframe=cellborder]
\prompt{In}{incolor}{21}{\boxspacing}
\begin{Verbatim}[commandchars=\\\{\}]
\PY{n}{A}\PY{o}{\PYZca{}}\PY{l+m}{2}    \PY{c+c1}{\PYZsh{} cell\PYZhy{}by\PYZhy{}cell operations}
\end{Verbatim}
\end{tcolorbox}

    \begin{tabular}{ll}
 a & b\\
\hline
	 1  & 16\\
	 4  & 25\\
	 9  & 36\\
\end{tabular}


    
    \begin{tcolorbox}[breakable, size=fbox, boxrule=1pt, pad at break*=1mm,colback=cellbackground, colframe=cellborder]
\prompt{In}{incolor}{22}{\boxspacing}
\begin{Verbatim}[commandchars=\\\{\}]
\PY{n}{A} \PY{o}{+} \PY{n+nf}{t}\PY{p}{(}\PY{n}{B}\PY{p}{)}
\end{Verbatim}
\end{tcolorbox}

    \begin{tabular}{ll}
 a & b\\
\hline
	 2  &  8\\
	 4  & 10\\
	 6  & 12\\
\end{tabular}


    
    \begin{tcolorbox}[breakable, size=fbox, boxrule=1pt, pad at break*=1mm,colback=cellbackground, colframe=cellborder]
\prompt{In}{incolor}{24}{\boxspacing}
\begin{Verbatim}[commandchars=\\\{\}]
\PY{c+c1}{\PYZsh{} standard matrix multiplication}

\PY{n}{A}\PY{o}{\PYZpc{}*\PYZpc{}}\PY{n}{B}  \PY{c+c1}{\PYZsh{} (3x2)*(2x3)}
\end{Verbatim}
\end{tcolorbox}

    \begin{tabular}{lll}
	 17 & 22 & 27\\
	 22 & 29 & 36\\
	 27 & 36 & 45\\
\end{tabular}


    
    \begin{tcolorbox}[breakable, size=fbox, boxrule=1pt, pad at break*=1mm,colback=cellbackground, colframe=cellborder]
\prompt{In}{incolor}{25}{\boxspacing}
\begin{Verbatim}[commandchars=\\\{\}]
\PY{c+c1}{\PYZsh{} B2.3. Lists}

\PY{n}{a\PYZus{}list} \PY{o}{\PYZlt{}\PYZhy{}} \PY{n+nf}{list}\PY{p}{(}\PY{n}{a}\PY{p}{,}\PY{n}{b}\PY{p}{)}
\PY{n}{a\PYZus{}list}
\end{Verbatim}
\end{tcolorbox}

    \begin{enumerate}
\item \begin{enumerate*}
\item 1
\item 2
\item 3
\end{enumerate*}

\item \begin{enumerate*}
\item 4
\item 5
\item 6
\end{enumerate*}

\end{enumerate}


    
    \begin{tcolorbox}[breakable, size=fbox, boxrule=1pt, pad at break*=1mm,colback=cellbackground, colframe=cellborder]
\prompt{In}{incolor}{26}{\boxspacing}
\begin{Verbatim}[commandchars=\\\{\}]
\PY{n}{b\PYZus{}list} \PY{o}{\PYZlt{}\PYZhy{}} \PY{n+nf}{list}\PY{p}{(}\PY{n}{a\PYZus{}list}\PY{p}{,}\PY{n}{A}\PY{p}{)}
\PY{n}{b\PYZus{}list}
\end{Verbatim}
\end{tcolorbox}

    \begin{enumerate}
\item \begin{enumerate}
\item \begin{enumerate*}
\item 1
\item 2
\item 3
\end{enumerate*}

\item \begin{enumerate*}
\item 4
\item 5
\item 6
\end{enumerate*}

\end{enumerate}

\item \begin{tabular}{ll}
 a & b\\
\hline
	 1 & 4\\
	 2 & 5\\
	 3 & 6\\
\end{tabular}

\end{enumerate}


    
    \begin{tcolorbox}[breakable, size=fbox, boxrule=1pt, pad at break*=1mm,colback=cellbackground, colframe=cellborder]
\prompt{In}{incolor}{27}{\boxspacing}
\begin{Verbatim}[commandchars=\\\{\}]
\PY{n}{c\PYZus{}list} \PY{o}{\PYZlt{}\PYZhy{}} \PY{n+nf}{list}\PY{p}{(}\PY{n}{A}\PY{p}{,}\PY{n}{B}\PY{p}{)}
\PY{n}{c\PYZus{}list}
\end{Verbatim}
\end{tcolorbox}

    \begin{enumerate}
\item \begin{tabular}{ll}
 a & b\\
\hline
	 1 & 4\\
	 2 & 5\\
	 3 & 6\\
\end{tabular}

\item \begin{tabular}{r|lll}
	a & 1 & 2 & 3\\
	b & 4 & 5 & 6\\
\end{tabular}

\end{enumerate}


    
    \begin{tcolorbox}[breakable, size=fbox, boxrule=1pt, pad at break*=1mm,colback=cellbackground, colframe=cellborder]
\prompt{In}{incolor}{28}{\boxspacing}
\begin{Verbatim}[commandchars=\\\{\}]
\PY{n}{c\PYZus{}list} \PY{o}{\PYZlt{}\PYZhy{}} \PY{n+nf}{c}\PY{p}{(}\PY{n+nf}{c}\PY{p}{(}\PY{l+s}{\PYZdq{}}\PY{l+s}{one\PYZdq{}}\PY{p}{,}\PY{l+s}{\PYZdq{}}\PY{l+s}{two\PYZdq{}}\PY{p}{)}\PY{p}{,}\PY{n}{c\PYZus{}list}\PY{p}{)}
\PY{n}{c\PYZus{}list}

\PY{c+c1}{\PYZsh{} operations of function c() looks similar to list(), but it is very much different}
\end{Verbatim}
\end{tcolorbox}

    \begin{enumerate}
\item 'one'
\item 'two'
\item \begin{tabular}{ll}
 a & b\\
\hline
	 1 & 4\\
	 2 & 5\\
	 3 & 6\\
\end{tabular}

\item \begin{tabular}{r|lll}
	a & 1 & 2 & 3\\
	b & 4 & 5 & 6\\
\end{tabular}

\end{enumerate}


    
    \begin{tcolorbox}[breakable, size=fbox, boxrule=1pt, pad at break*=1mm,colback=cellbackground, colframe=cellborder]
\prompt{In}{incolor}{30}{\boxspacing}
\begin{Verbatim}[commandchars=\\\{\}]
\PY{c+c1}{\PYZsh{} B3 Interacting with Objects}

\PY{c+c1}{\PYZsh{} B3.1. Transforming Objects}

\PY{n}{B}
\PY{n+nf}{as.vector}\PY{p}{(}\PY{n}{B}\PY{p}{)}
\end{Verbatim}
\end{tcolorbox}

    \begin{tabular}{r|lll}
	a & 1 & 2 & 3\\
	b & 4 & 5 & 6\\
\end{tabular}


    
    \begin{enumerate*}
\item 1
\item 4
\item 2
\item 5
\item 3
\item 6
\end{enumerate*}


    
    \begin{tcolorbox}[breakable, size=fbox, boxrule=1pt, pad at break*=1mm,colback=cellbackground, colframe=cellborder]
\prompt{In}{incolor}{32}{\boxspacing}
\begin{Verbatim}[commandchars=\\\{\}]
\PY{n}{a}
\PY{n}{b}
\PY{n+nf}{c}\PY{p}{(}\PY{n}{a}\PY{p}{,}\PY{n}{b}\PY{p}{)}
\PY{n+nf}{matrix}\PY{p}{(}\PY{n+nf}{c}\PY{p}{(}\PY{n}{a}\PY{p}{,}\PY{n}{b}\PY{p}{)}\PY{p}{,}\PY{n}{nrow}\PY{o}{=}\PY{l+m}{3}\PY{p}{)}
\end{Verbatim}
\end{tcolorbox}

    \begin{enumerate*}
\item 1
\item 2
\item 3
\end{enumerate*}


    
    \begin{enumerate*}
\item 4
\item 5
\item 6
\end{enumerate*}


    
    \begin{enumerate*}
\item 1
\item 2
\item 3
\item 4
\item 5
\item 6
\end{enumerate*}


    
    \begin{tabular}{ll}
	 1 & 4\\
	 2 & 5\\
	 3 & 6\\
\end{tabular}


    
    \begin{tcolorbox}[breakable, size=fbox, boxrule=1pt, pad at break*=1mm,colback=cellbackground, colframe=cellborder]
\prompt{In}{incolor}{34}{\boxspacing}
\begin{Verbatim}[commandchars=\\\{\}]
\PY{n+nf}{cbind}\PY{p}{(}\PY{n}{a}\PY{p}{,}\PY{n}{b}\PY{p}{)}
\PY{n+nf}{as.matrix}\PY{p}{(}\PY{n+nf}{cbind}\PY{p}{(}\PY{n}{a}\PY{p}{,}\PY{n}{b}\PY{p}{)}\PY{p}{)}
\end{Verbatim}
\end{tcolorbox}

    \begin{tabular}{ll}
 a & b\\
\hline
	 1 & 4\\
	 2 & 5\\
	 3 & 6\\
\end{tabular}


    
    \begin{tabular}{ll}
 a & b\\
\hline
	 1 & 4\\
	 2 & 5\\
	 3 & 6\\
\end{tabular}


    
    \begin{tcolorbox}[breakable, size=fbox, boxrule=1pt, pad at break*=1mm,colback=cellbackground, colframe=cellborder]
\prompt{In}{incolor}{36}{\boxspacing}
\begin{Verbatim}[commandchars=\\\{\}]
\PY{n}{B}
\PY{n+nf}{as.list}\PY{p}{(}\PY{n}{B}\PY{p}{)}
\end{Verbatim}
\end{tcolorbox}

    \begin{tabular}{r|lll}
	a & 1 & 2 & 3\\
	b & 4 & 5 & 6\\
\end{tabular}


    
    \begin{enumerate}
\item 1
\item 4
\item 2
\item 5
\item 3
\item 6
\end{enumerate}


    
    \begin{tcolorbox}[breakable, size=fbox, boxrule=1pt, pad at break*=1mm,colback=cellbackground, colframe=cellborder]
\prompt{In}{incolor}{37}{\boxspacing}
\begin{Verbatim}[commandchars=\\\{\}]
\PY{n}{a\PYZus{}list}
\PY{n+nf}{unlist}\PY{p}{(}\PY{n}{a\PYZus{}list}\PY{p}{)}
\end{Verbatim}
\end{tcolorbox}

    \begin{enumerate}
\item \begin{enumerate*}
\item 1
\item 2
\item 3
\end{enumerate*}

\item \begin{enumerate*}
\item 4
\item 5
\item 6
\end{enumerate*}

\end{enumerate}


    
    \begin{enumerate*}
\item 1
\item 2
\item 3
\item 4
\item 5
\item 6
\end{enumerate*}


    
    \begin{tcolorbox}[breakable, size=fbox, boxrule=1pt, pad at break*=1mm,colback=cellbackground, colframe=cellborder]
\prompt{In}{incolor}{38}{\boxspacing}
\begin{Verbatim}[commandchars=\\\{\}]
\PY{n}{A}
\PY{n+nf}{as.character}\PY{p}{(}\PY{n}{A}\PY{p}{)}
\end{Verbatim}
\end{tcolorbox}

    \begin{tabular}{ll}
 a & b\\
\hline
	 1 & 4\\
	 2 & 5\\
	 3 & 6\\
\end{tabular}


    
    \begin{enumerate*}
\item '1'
\item '2'
\item '3'
\item '4'
\item '5'
\item '6'
\end{enumerate*}


    
    \begin{tcolorbox}[breakable, size=fbox, boxrule=1pt, pad at break*=1mm,colback=cellbackground, colframe=cellborder]
\prompt{In}{incolor}{39}{\boxspacing}
\begin{Verbatim}[commandchars=\\\{\}]
\PY{n}{B}
\PY{n+nf}{as.factor}\PY{p}{(}\PY{n}{B}\PY{p}{)}
\end{Verbatim}
\end{tcolorbox}

    \begin{tabular}{r|lll}
	a & 1 & 2 & 3\\
	b & 4 & 5 & 6\\
\end{tabular}


    
    \begin{enumerate*}
\item 1
\item 4
\item 2
\item 5
\item 3
\item 6
\end{enumerate*}

\emph{Levels}: \begin{enumerate*}
\item '1'
\item '2'
\item '3'
\item '4'
\item '5'
\item '6'
\end{enumerate*}


    
    \begin{tcolorbox}[breakable, size=fbox, boxrule=1pt, pad at break*=1mm,colback=cellbackground, colframe=cellborder]
\prompt{In}{incolor}{41}{\boxspacing}
\begin{Verbatim}[commandchars=\\\{\}]
\PY{n}{a}
\PY{n+nf}{as.vector}\PY{p}{(}\PY{n+nf}{as.numeric}\PY{p}{(}\PY{n+nf}{as.character}\PY{p}{(}\PY{n+nf}{as.factor}\PY{p}{(}\PY{n}{a}\PY{p}{)}\PY{p}{)}\PY{p}{)}\PY{p}{)} \PY{o}{==} \PY{n}{a}
\end{Verbatim}
\end{tcolorbox}

    \begin{enumerate*}
\item 1
\item 2
\item 3
\end{enumerate*}


    
    \begin{enumerate*}
\item TRUE
\item TRUE
\item TRUE
\end{enumerate*}


    
    \begin{tcolorbox}[breakable, size=fbox, boxrule=1pt, pad at break*=1mm,colback=cellbackground, colframe=cellborder]
\prompt{In}{incolor}{42}{\boxspacing}
\begin{Verbatim}[commandchars=\\\{\}]
\PY{c+c1}{\PYZsh{} B3.2. Logical Expressions}

\PY{n}{a}
\PY{n}{b}
\PY{n}{a} \PY{o}{==} \PY{n}{b}
\end{Verbatim}
\end{tcolorbox}

    \begin{enumerate*}
\item 1
\item 2
\item 3
\end{enumerate*}


    
    \begin{enumerate*}
\item 4
\item 5
\item 6
\end{enumerate*}


    
    \begin{enumerate*}
\item FALSE
\item FALSE
\item FALSE
\end{enumerate*}


    
    \begin{tcolorbox}[breakable, size=fbox, boxrule=1pt, pad at break*=1mm,colback=cellbackground, colframe=cellborder]
\prompt{In}{incolor}{43}{\boxspacing}
\begin{Verbatim}[commandchars=\\\{\}]
\PY{n}{A}
\PY{n}{B}
\PY{n}{A} \PY{o}{==} \PY{n+nf}{t}\PY{p}{(}\PY{n}{B}\PY{p}{)}
\end{Verbatim}
\end{tcolorbox}

    \begin{tabular}{ll}
 a & b\\
\hline
	 1 & 4\\
	 2 & 5\\
	 3 & 6\\
\end{tabular}


    
    \begin{tabular}{r|lll}
	a & 1 & 2 & 3\\
	b & 4 & 5 & 6\\
\end{tabular}


    
    \begin{tabular}{ll}
 a & b\\
\hline
	 TRUE & TRUE\\
	 TRUE & TRUE\\
	 TRUE & TRUE\\
\end{tabular}


    
    \begin{tcolorbox}[breakable, size=fbox, boxrule=1pt, pad at break*=1mm,colback=cellbackground, colframe=cellborder]
\prompt{In}{incolor}{45}{\boxspacing}
\begin{Verbatim}[commandchars=\\\{\}]
\PY{n}{a\PYZus{}list}
\PY{n}{a\PYZus{}list}\PY{p}{[[}\PY{l+m}{1}\PY{p}{]]}
\PY{n}{a\PYZus{}list}\PY{p}{[[}\PY{l+m}{2}\PY{p}{]]}
\PY{n}{a\PYZus{}list}\PY{p}{[[}\PY{l+m}{1}\PY{p}{]]} \PY{o}{==} \PY{n}{a\PYZus{}list}\PY{p}{[[}\PY{l+m}{2}\PY{p}{]]}
\end{Verbatim}
\end{tcolorbox}

    \begin{enumerate}
\item \begin{enumerate*}
\item 1
\item 2
\item 3
\end{enumerate*}

\item \begin{enumerate*}
\item 4
\item 5
\item 6
\end{enumerate*}

\end{enumerate}


    
    \begin{enumerate*}
\item 1
\item 2
\item 3
\end{enumerate*}


    
    \begin{enumerate*}
\item 4
\item 5
\item 6
\end{enumerate*}


    
    \begin{enumerate*}
\item FALSE
\item FALSE
\item FALSE
\end{enumerate*}


    
    \begin{tcolorbox}[breakable, size=fbox, boxrule=1pt, pad at break*=1mm,colback=cellbackground, colframe=cellborder]
\prompt{In}{incolor}{47}{\boxspacing}
\begin{Verbatim}[commandchars=\\\{\}]
\PY{n}{a}
\PY{n}{b}

\PY{n}{a} \PY{o}{\PYZgt{}} \PY{n}{b}
\end{Verbatim}
\end{tcolorbox}

    \begin{enumerate*}
\item 1
\item 2
\item 3
\end{enumerate*}


    
    \begin{enumerate*}
\item 4
\item 5
\item 6
\end{enumerate*}


    
    \begin{enumerate*}
\item FALSE
\item FALSE
\item FALSE
\end{enumerate*}


    
    \begin{tcolorbox}[breakable, size=fbox, boxrule=1pt, pad at break*=1mm,colback=cellbackground, colframe=cellborder]
\prompt{In}{incolor}{48}{\boxspacing}
\begin{Verbatim}[commandchars=\\\{\}]
\PY{n}{b} \PY{o}{\PYZgt{}} \PY{l+m}{5}
\end{Verbatim}
\end{tcolorbox}

    \begin{enumerate*}
\item FALSE
\item FALSE
\item TRUE
\end{enumerate*}


    
    \begin{tcolorbox}[breakable, size=fbox, boxrule=1pt, pad at break*=1mm,colback=cellbackground, colframe=cellborder]
\prompt{In}{incolor}{49}{\boxspacing}
\begin{Verbatim}[commandchars=\\\{\}]
\PY{n}{b} \PY{o}{\PYZgt{}=} \PY{l+m}{5}
\end{Verbatim}
\end{tcolorbox}

    \begin{enumerate*}
\item FALSE
\item TRUE
\item TRUE
\end{enumerate*}


    
    \begin{tcolorbox}[breakable, size=fbox, boxrule=1pt, pad at break*=1mm,colback=cellbackground, colframe=cellborder]
\prompt{In}{incolor}{50}{\boxspacing}
\begin{Verbatim}[commandchars=\\\{\}]
\PY{n}{b} \PY{o}{\PYZlt{}=} \PY{l+m}{4}
\end{Verbatim}
\end{tcolorbox}

    \begin{enumerate*}
\item TRUE
\item FALSE
\item FALSE
\end{enumerate*}


    
    \begin{tcolorbox}[breakable, size=fbox, boxrule=1pt, pad at break*=1mm,colback=cellbackground, colframe=cellborder]
\prompt{In}{incolor}{51}{\boxspacing}
\begin{Verbatim}[commandchars=\\\{\}]
\PY{n}{a} \PY{o}{!=} \PY{n}{b}
\end{Verbatim}
\end{tcolorbox}

    \begin{enumerate*}
\item TRUE
\item TRUE
\item TRUE
\end{enumerate*}


    
    \begin{tcolorbox}[breakable, size=fbox, boxrule=1pt, pad at break*=1mm,colback=cellbackground, colframe=cellborder]
\prompt{In}{incolor}{52}{\boxspacing}
\begin{Verbatim}[commandchars=\\\{\}]
\PY{p}{(}\PY{n}{b} \PY{o}{\PYZgt{}} \PY{l+m}{4}\PY{p}{)} \PY{o}{\PYZam{}} \PY{n}{a} \PY{o}{==} \PY{l+m}{3}
\end{Verbatim}
\end{tcolorbox}

    \begin{enumerate*}
\item FALSE
\item FALSE
\item TRUE
\end{enumerate*}


    
    \begin{tcolorbox}[breakable, size=fbox, boxrule=1pt, pad at break*=1mm,colback=cellbackground, colframe=cellborder]
\prompt{In}{incolor}{54}{\boxspacing}
\begin{Verbatim}[commandchars=\\\{\}]
\PY{p}{(}\PY{n}{b} \PY{o}{\PYZgt{}} \PY{l+m}{4}\PY{p}{)} \PY{o}{\PYZam{}\PYZam{}} \PY{n}{a} \PY{o}{==} \PY{l+m}{3}    \PY{c+c1}{\PYZsh{} \PYZam{}\PYZam{} asks whether all the element is the same}
\end{Verbatim}
\end{tcolorbox}

    FALSE

    
    \begin{tcolorbox}[breakable, size=fbox, boxrule=1pt, pad at break*=1mm,colback=cellbackground, colframe=cellborder]
\prompt{In}{incolor}{55}{\boxspacing}
\begin{Verbatim}[commandchars=\\\{\}]
\PY{p}{(}\PY{n}{b} \PY{o}{\PYZgt{}} \PY{l+m}{4}\PY{p}{)} \PY{o}{|} \PY{n}{a} \PY{o}{==} \PY{l+m}{3}
\end{Verbatim}
\end{tcolorbox}

    \begin{enumerate*}
\item FALSE
\item TRUE
\item TRUE
\end{enumerate*}


    
    \begin{tcolorbox}[breakable, size=fbox, boxrule=1pt, pad at break*=1mm,colback=cellbackground, colframe=cellborder]
\prompt{In}{incolor}{56}{\boxspacing}
\begin{Verbatim}[commandchars=\\\{\}]
\PY{p}{(}\PY{n}{b} \PY{o}{\PYZgt{}} \PY{l+m}{4}\PY{p}{)} \PY{o}{||} \PY{n}{a} \PY{o}{==} \PY{l+m}{3}   \PY{c+c1}{\PYZsh{} || asks whether all the element is the same}
\end{Verbatim}
\end{tcolorbox}

    FALSE

    
    \begin{tcolorbox}[breakable, size=fbox, boxrule=1pt, pad at break*=1mm,colback=cellbackground, colframe=cellborder]
\prompt{In}{incolor}{57}{\boxspacing}
\begin{Verbatim}[commandchars=\\\{\}]
\PY{c+c1}{\PYZsh{} B3.3. Retrieving Information from a Position}

\PY{n}{a}
\PY{n}{a}\PY{p}{[}\PY{l+m}{1}\PY{p}{]}
\end{Verbatim}
\end{tcolorbox}

    \begin{enumerate*}
\item 1
\item 2
\item 3
\end{enumerate*}


    
    1

    
    \begin{tcolorbox}[breakable, size=fbox, boxrule=1pt, pad at break*=1mm,colback=cellbackground, colframe=cellborder]
\prompt{In}{incolor}{58}{\boxspacing}
\begin{Verbatim}[commandchars=\\\{\}]
\PY{n}{b}
\PY{n}{b}\PY{p}{[}\PY{l+m}{3}\PY{p}{]}
\end{Verbatim}
\end{tcolorbox}

    \begin{enumerate*}
\item 4
\item 5
\item 6
\end{enumerate*}


    
    6

    
    \begin{tcolorbox}[breakable, size=fbox, boxrule=1pt, pad at break*=1mm,colback=cellbackground, colframe=cellborder]
\prompt{In}{incolor}{59}{\boxspacing}
\begin{Verbatim}[commandchars=\\\{\}]
\PY{n}{A}
\PY{n}{A}\PY{p}{[}\PY{l+m}{5}\PY{p}{]}
\end{Verbatim}
\end{tcolorbox}

    \begin{tabular}{ll}
 a & b\\
\hline
	 1 & 4\\
	 2 & 5\\
	 3 & 6\\
\end{tabular}


    
    5

    
    \begin{tcolorbox}[breakable, size=fbox, boxrule=1pt, pad at break*=1mm,colback=cellbackground, colframe=cellborder]
\prompt{In}{incolor}{61}{\boxspacing}
\begin{Verbatim}[commandchars=\\\{\}]
\PY{n}{A}
\PY{n}{A}\PY{p}{[}\PY{l+m}{2}\PY{p}{,}\PY{l+m}{2}\PY{p}{]}
\end{Verbatim}
\end{tcolorbox}

    \begin{tabular}{ll}
 a & b\\
\hline
	 1 & 4\\
	 2 & 5\\
	 3 & 6\\
\end{tabular}


    
    \textbf{b:} 5

    
    \begin{tcolorbox}[breakable, size=fbox, boxrule=1pt, pad at break*=1mm,colback=cellbackground, colframe=cellborder]
\prompt{In}{incolor}{62}{\boxspacing}
\begin{Verbatim}[commandchars=\\\{\}]
\PY{n}{a}
\PY{n}{a}\PY{p}{[}\PY{l+m}{1}\PY{o}{:}\PY{l+m}{2}\PY{p}{]}
\end{Verbatim}
\end{tcolorbox}

    \begin{enumerate*}
\item 1
\item 2
\item 3
\end{enumerate*}


    
    \begin{enumerate*}
\item 1
\item 2
\end{enumerate*}


    
    \begin{tcolorbox}[breakable, size=fbox, boxrule=1pt, pad at break*=1mm,colback=cellbackground, colframe=cellborder]
\prompt{In}{incolor}{63}{\boxspacing}
\begin{Verbatim}[commandchars=\\\{\}]
\PY{n}{a}\PY{p}{[}\PY{l+m}{\PYZhy{}3}\PY{p}{]}
\end{Verbatim}
\end{tcolorbox}

    \begin{enumerate*}
\item 1
\item 2
\end{enumerate*}


    
    \begin{tcolorbox}[breakable, size=fbox, boxrule=1pt, pad at break*=1mm,colback=cellbackground, colframe=cellborder]
\prompt{In}{incolor}{64}{\boxspacing}
\begin{Verbatim}[commandchars=\\\{\}]
\PY{n}{A}
\PY{n}{A}\PY{p}{[}\PY{l+m}{1}\PY{p}{,}\PY{p}{]}   \PY{c+c1}{\PYZsh{} take row 1, for all column}
\end{Verbatim}
\end{tcolorbox}

    \begin{tabular}{ll}
 a & b\\
\hline
	 1 & 4\\
	 2 & 5\\
	 3 & 6\\
\end{tabular}


    
    \begin{description*}
\item[a] 1
\item[b] 4
\end{description*}


    
    \begin{tcolorbox}[breakable, size=fbox, boxrule=1pt, pad at break*=1mm,colback=cellbackground, colframe=cellborder]
\prompt{In}{incolor}{65}{\boxspacing}
\begin{Verbatim}[commandchars=\\\{\}]
\PY{n}{A}
\PY{n}{A}\PY{p}{[}\PY{n+nf}{c}\PY{p}{(}\PY{l+m}{1}\PY{p}{,}\PY{l+m}{5}\PY{p}{)}\PY{p}{]}
\end{Verbatim}
\end{tcolorbox}

    \begin{tabular}{ll}
 a & b\\
\hline
	 1 & 4\\
	 2 & 5\\
	 3 & 6\\
\end{tabular}


    
    \begin{enumerate*}
\item 1
\item 5
\end{enumerate*}


    
    \begin{tcolorbox}[breakable, size=fbox, boxrule=1pt, pad at break*=1mm,colback=cellbackground, colframe=cellborder]
\prompt{In}{incolor}{67}{\boxspacing}
\begin{Verbatim}[commandchars=\\\{\}]
\PY{n}{a}
\PY{n+nf}{length}\PY{p}{(}\PY{n}{a}\PY{p}{)}
\PY{n}{a}\PY{p}{[}\PY{l+m}{2}\PY{o}{:}\PY{n+nf}{length}\PY{p}{(}\PY{n}{a}\PY{p}{)}\PY{p}{]}
\end{Verbatim}
\end{tcolorbox}

    \begin{enumerate*}
\item 1
\item 2
\item 3
\end{enumerate*}


    
    3

    
    \begin{enumerate*}
\item 2
\item 3
\end{enumerate*}


    
    \begin{tcolorbox}[breakable, size=fbox, boxrule=1pt, pad at break*=1mm,colback=cellbackground, colframe=cellborder]
\prompt{In}{incolor}{68}{\boxspacing}
\begin{Verbatim}[commandchars=\\\{\}]
\PY{n}{A}
\PY{n}{D} \PY{o}{\PYZlt{}\PYZhy{}} \PY{n+nf}{cbind}\PY{p}{(}\PY{n}{A}\PY{p}{,} \PY{l+m}{2}\PY{o}{*}\PY{n}{A}\PY{p}{)}
\PY{n}{D}
\end{Verbatim}
\end{tcolorbox}

    \begin{tabular}{ll}
 a & b\\
\hline
	 1 & 4\\
	 2 & 5\\
	 3 & 6\\
\end{tabular}


    
    \begin{tabular}{llll}
 a & b & a & b\\
\hline
	 1  & 4  & 2  &  8\\
	 2  & 5  & 4  & 10\\
	 3  & 6  & 6  & 12\\
\end{tabular}


    
    \begin{tcolorbox}[breakable, size=fbox, boxrule=1pt, pad at break*=1mm,colback=cellbackground, colframe=cellborder]
\prompt{In}{incolor}{69}{\boxspacing}
\begin{Verbatim}[commandchars=\\\{\}]
\PY{n}{D}
\PY{n+nf}{dim}\PY{p}{(}\PY{n}{D}\PY{p}{)}           \PY{c+c1}{\PYZsh{} dimension: 3x4}
\PY{n+nf}{dim}\PY{p}{(}\PY{n}{D}\PY{p}{)}\PY{p}{[}\PY{l+m}{2}\PY{p}{]}        \PY{c+c1}{\PYZsh{} take the column as the \PYZsq{}position\PYZsq{}, that is 4}
\PY{n}{D}\PY{p}{[}\PY{p}{,}\PY{l+m}{3}\PY{o}{:}\PY{n+nf}{dim}\PY{p}{(}\PY{n}{D}\PY{p}{)}\PY{p}{[}\PY{l+m}{2}\PY{p}{]]}  \PY{c+c1}{\PYZsh{} take all row for each column starting from 3 to 4 (dim(D)[2])}
\end{Verbatim}
\end{tcolorbox}

    \begin{tabular}{llll}
 a & b & a & b\\
\hline
	 1  & 4  & 2  &  8\\
	 2  & 5  & 4  & 10\\
	 3  & 6  & 6  & 12\\
\end{tabular}


    
    \begin{enumerate*}
\item 3
\item 4
\end{enumerate*}


    
    4

    
    \begin{tabular}{ll}
 a & b\\
\hline
	 2  &  8\\
	 4  & 10\\
	 6  & 12\\
\end{tabular}


    
    \begin{tcolorbox}[breakable, size=fbox, boxrule=1pt, pad at break*=1mm,colback=cellbackground, colframe=cellborder]
\prompt{In}{incolor}{75}{\boxspacing}
\begin{Verbatim}[commandchars=\\\{\}]
\PY{n}{a\PYZus{}list}
\PY{n}{a\PYZus{}list}\PY{p}{[}\PY{l+m}{2}\PY{p}{]}

\PY{n}{a\PYZus{}list}\PY{p}{[}\PY{l+m}{2}\PY{p}{]}\PY{p}{[}\PY{l+m}{2}\PY{p}{]}   \PY{c+c1}{\PYZsh{} no list, since a\PYZus{}list[2] just consist of one list. there is no 2nd list in this particular a\PYZus{}list[2] list}


\PY{n}{a\PYZus{}list}\PY{p}{[[}\PY{l+m}{2}\PY{p}{]]}    \PY{c+c1}{\PYZsh{} taking what is inside the second list from a\PYZus{}list. resulting in a vector\PYZhy{}like object}

\PY{n}{a\PYZus{}list}\PY{p}{[[}\PY{l+m}{2}\PY{p}{]]}\PY{p}{[}\PY{l+m}{2}\PY{p}{]} \PY{c+c1}{\PYZsh{} first, taking what is inside the second list from a\PYZus{}list. then, take the second element from it}
\end{Verbatim}
\end{tcolorbox}

    \begin{enumerate}
\item \begin{enumerate*}
\item 1
\item 2
\item 3
\end{enumerate*}

\item \begin{enumerate*}
\item 4
\item 5
\item 6
\end{enumerate*}

\end{enumerate}


    
    \begin{enumerate}
\item \begin{enumerate*}
\item 4
\item 5
\item 6
\end{enumerate*}

\end{enumerate}


    
    \begin{enumerate}
\item NULL
\end{enumerate}


    
    \begin{enumerate*}
\item 4
\item 5
\item 6
\end{enumerate*}


    
    5

    
    \begin{tcolorbox}[breakable, size=fbox, boxrule=1pt, pad at break*=1mm,colback=cellbackground, colframe=cellborder]
\prompt{In}{incolor}{78}{\boxspacing}
\begin{Verbatim}[commandchars=\\\{\}]
\PY{n}{a\PYZus{}list}

\PY{n+nf}{names}\PY{p}{(}\PY{n}{a\PYZus{}list}\PY{p}{)} \PY{o}{\PYZlt{}\PYZhy{}} \PY{n+nf}{c}\PY{p}{(}\PY{l+s}{\PYZdq{}}\PY{l+s}{first\PYZdq{}}\PY{p}{,}\PY{l+s}{\PYZdq{}}\PY{l+s}{second\PYZdq{}}\PY{p}{)}

\PY{n+nf}{names}\PY{p}{(}\PY{n}{a\PYZus{}list}\PY{p}{)}

\PY{n}{a\PYZus{}list}\PY{o}{\PYZdl{}}\PY{n}{first}    \PY{c+c1}{\PYZsh{} \PYZdl{} retrieves a named item in a list}

\PY{n+nf}{names}\PY{p}{(}\PY{n}{a\PYZus{}list}\PY{p}{)}\PY{p}{[}\PY{l+m}{2}\PY{p}{]}
\end{Verbatim}
\end{tcolorbox}

    \begin{description}
\item[\$first] \begin{enumerate*}
\item 1
\item 2
\item 3
\end{enumerate*}

\item[\$second] \begin{enumerate*}
\item 4
\item 5
\item 6
\end{enumerate*}

\end{description}


    
    \begin{enumerate*}
\item 'first'
\item 'second'
\end{enumerate*}


    
    \begin{enumerate*}
\item 1
\item 2
\item 3
\end{enumerate*}


    
    'second'

    
    \begin{tcolorbox}[breakable, size=fbox, boxrule=1pt, pad at break*=1mm,colback=cellbackground, colframe=cellborder]
\prompt{In}{incolor}{82}{\boxspacing}
\begin{Verbatim}[commandchars=\\\{\}]
\PY{n}{a\PYZus{}list}
\PY{n}{B}

\PY{n}{b\PYZus{}list} \PY{o}{\PYZlt{}\PYZhy{}} \PY{n+nf}{list}\PY{p}{(}\PY{n}{a\PYZus{}list}\PY{p}{,}\PY{n}{B}\PY{o}{=}\PY{n}{B}\PY{p}{)}

\PY{n}{b\PYZus{}list}

\PY{n}{b\PYZus{}list}\PY{o}{\PYZdl{}}\PY{n}{B}
\end{Verbatim}
\end{tcolorbox}

    \begin{description}
\item[\$first] \begin{enumerate*}
\item 1
\item 2
\item 3
\end{enumerate*}

\item[\$second] \begin{enumerate*}
\item 4
\item 5
\item 6
\end{enumerate*}

\end{description}


    
    \begin{tabular}{r|lll}
	a & 1 & 2 & 3\\
	b & 4 & 5 & 6\\
\end{tabular}


    
    \begin{description}
\item[{[[1]]}] \begin{description}
\item[\$first] \begin{enumerate*}
\item 1
\item 2
\item 3
\end{enumerate*}

\item[\$second] \begin{enumerate*}
\item 4
\item 5
\item 6
\end{enumerate*}

\end{description}

\item[\$B] \begin{tabular}{r|lll}
	a & 1 & 2 & 3\\
	b & 4 & 5 & 6\\
\end{tabular}

\end{description}


    
    \begin{tabular}{r|lll}
	a & 1 & 2 & 3\\
	b & 4 & 5 & 6\\
\end{tabular}


    
    \begin{tcolorbox}[breakable, size=fbox, boxrule=1pt, pad at break*=1mm,colback=cellbackground, colframe=cellborder]
\prompt{In}{incolor}{83}{\boxspacing}
\begin{Verbatim}[commandchars=\\\{\}]
\PY{c+c1}{\PYZsh{} B3.4 Retrieving the Position from the Information}

\PY{n}{a}

\PY{n+nf}{which}\PY{p}{(}\PY{n}{a} \PY{o}{\PYZgt{}} \PY{l+m}{2}\PY{p}{)}
\end{Verbatim}
\end{tcolorbox}

    \begin{enumerate*}
\item 1
\item 2
\item 3
\end{enumerate*}


    
    3

    
    \begin{tcolorbox}[breakable, size=fbox, boxrule=1pt, pad at break*=1mm,colback=cellbackground, colframe=cellborder]
\prompt{In}{incolor}{84}{\boxspacing}
\begin{Verbatim}[commandchars=\\\{\}]
\PY{n}{A}

\PY{n+nf}{which}\PY{p}{(}\PY{n}{A} \PY{o}{\PYZgt{}} \PY{l+m}{2}\PY{p}{)}             \PY{c+c1}{\PYZsh{} it retrieves the position in the matrix A that is higher than 2}
                         \PY{c+c1}{\PYZsh{} in our case, (3,1), (1,2), (2,2), and (3,2) are indeed larger than 2}
                         \PY{c+c1}{\PYZsh{} (3,1) == 3, (1,2) == 4, (2,2) == 5, (3,2) == 6}
\end{Verbatim}
\end{tcolorbox}

    \begin{tabular}{ll}
 a & b\\
\hline
	 1 & 4\\
	 2 & 5\\
	 3 & 6\\
\end{tabular}


    
    \begin{enumerate*}
\item 3
\item 4
\item 5
\item 6
\end{enumerate*}


    
    \begin{tcolorbox}[breakable, size=fbox, boxrule=1pt, pad at break*=1mm,colback=cellbackground, colframe=cellborder]
\prompt{In}{incolor}{88}{\boxspacing}
\begin{Verbatim}[commandchars=\\\{\}]
\PY{n}{a\PYZus{}list}

\PY{n}{a\PYZus{}list}\PY{p}{[[}\PY{l+m}{2}\PY{p}{]]}

\PY{n+nf}{which}\PY{p}{(}\PY{n}{a\PYZus{}list}\PY{p}{[[}\PY{l+m}{2}\PY{p}{]]} \PY{o}{\PYZgt{}} \PY{l+m}{2}\PY{p}{)}     \PY{c+c1}{\PYZsh{} it retrieves the position in the vector a\PYZus{}list that is higher than 2. }
                           \PY{c+c1}{\PYZsh{} in our case, elements 1,2, and 3 are indeed larger than 2}
\end{Verbatim}
\end{tcolorbox}

    \begin{description}
\item[\$first] \begin{enumerate*}
\item 1
\item 2
\item 3
\end{enumerate*}

\item[\$second] \begin{enumerate*}
\item 4
\item 5
\item 6
\end{enumerate*}

\end{description}


    
    \begin{enumerate*}
\item 4
\item 5
\item 6
\end{enumerate*}


    
    \begin{enumerate*}
\item 1
\item 2
\item 3
\end{enumerate*}


    
    \begin{tcolorbox}[breakable, size=fbox, boxrule=1pt, pad at break*=1mm,colback=cellbackground, colframe=cellborder]
\prompt{In}{incolor}{105}{\boxspacing}
\begin{Verbatim}[commandchars=\\\{\}]
\PY{n}{b}

\PY{n}{a} \PY{o}{\PYZgt{}} \PY{l+m}{2}

\PY{n}{b}\PY{p}{[}\PY{n}{a} \PY{o}{\PYZgt{}} \PY{l+m}{2}\PY{p}{]}   \PY{c+c1}{\PYZsh{} when a is greater than 2? it is in position/index 3, then take element 3. we have b[3]. b[3] returns 6}
\end{Verbatim}
\end{tcolorbox}

    \begin{enumerate*}
\item 4
\item 5
\item 6
\end{enumerate*}


    
    \begin{enumerate*}
\item FALSE
\item FALSE
\item TRUE
\end{enumerate*}


    
    6

    
    \begin{tcolorbox}[breakable, size=fbox, boxrule=1pt, pad at break*=1mm,colback=cellbackground, colframe=cellborder]
\prompt{In}{incolor}{109}{\boxspacing}
\begin{Verbatim}[commandchars=\\\{\}]
\PY{n}{B}

\PY{n}{A} \PY{o}{\PYZgt{}} \PY{l+m}{2}

\PY{n}{B}\PY{p}{[}\PY{l+m}{3}\PY{p}{]}

\PY{n}{B}\PY{p}{[}\PY{n}{A} \PY{o}{\PYZgt{}} \PY{l+m}{2}\PY{p}{]}  \PY{c+c1}{\PYZsh{} when A is greater than 2? it is in position/index 3,4,5,6 (start for each column, compute position for all row). }
          \PY{c+c1}{\PYZsh{} Then, it returns 2,5,3,6}
\end{Verbatim}
\end{tcolorbox}

    \begin{tabular}{r|lll}
	a & 1 & 2 & 3\\
	b & 4 & 5 & 6\\
\end{tabular}


    
    \begin{tabular}{ll}
 a & b\\
\hline
	 FALSE & TRUE \\
	 FALSE & TRUE \\
	  TRUE & TRUE \\
\end{tabular}


    
    2

    
    \begin{enumerate*}
\item 2
\item 5
\item 3
\item 6
\end{enumerate*}


    
    \begin{tcolorbox}[breakable, size=fbox, boxrule=1pt, pad at break*=1mm,colback=cellbackground, colframe=cellborder]
\prompt{In}{incolor}{111}{\boxspacing}
\begin{Verbatim}[commandchars=\\\{\}]
\PY{n}{A}

\PY{n+nf}{colnames}\PY{p}{(}\PY{n}{A}\PY{p}{)}

\PY{n}{A}\PY{p}{[}\PY{p}{,} \PY{n+nf}{colnames}\PY{p}{(}\PY{n}{A}\PY{p}{)}\PY{o}{==}\PY{l+s}{\PYZdq{}}\PY{l+s}{b\PYZdq{}}\PY{p}{]}    \PY{c+c1}{\PYZsh{} retrieves every row for which the column name is \PYZdq{}b\PYZdq{}. then you have a vector c(4,5,6)}
\end{Verbatim}
\end{tcolorbox}

    \begin{tabular}{ll}
 a & b\\
\hline
	 1 & 4\\
	 2 & 5\\
	 3 & 6\\
\end{tabular}


    
    \begin{enumerate*}
\item 'a'
\item 'b'
\end{enumerate*}


    
    \begin{enumerate*}
\item 4
\item 5
\item 6
\end{enumerate*}


    
    \begin{tcolorbox}[breakable, size=fbox, boxrule=1pt, pad at break*=1mm,colback=cellbackground, colframe=cellborder]
\prompt{In}{incolor}{113}{\boxspacing}
\begin{Verbatim}[commandchars=\\\{\}]
\PY{c+c1}{\PYZsh{} on match() and \PYZpc{}in\PYZpc{}}

\PY{n}{d} \PY{o}{\PYZlt{}\PYZhy{}} \PY{n+nf}{c}\PY{p}{(}\PY{l+s}{\PYZdq{}}\PY{l+s}{one\PYZdq{}}\PY{p}{,}\PY{l+s}{\PYZdq{}}\PY{l+s}{two\PYZdq{}}\PY{p}{,}\PY{l+s}{\PYZdq{}}\PY{l+s}{three\PYZdq{}}\PY{p}{,}\PY{l+s}{\PYZdq{}}\PY{l+s}{four\PYZdq{}}\PY{p}{)}
\PY{n+nf}{c}\PY{p}{(}\PY{l+s}{\PYZdq{}}\PY{l+s}{one\PYZdq{}}\PY{p}{,}\PY{l+s}{\PYZdq{}}\PY{l+s}{four\PYZdq{}}\PY{p}{,}\PY{l+s}{\PYZdq{}}\PY{l+s}{five\PYZdq{}}\PY{p}{)} \PY{o}{\PYZpc{}in\PYZpc{}} \PY{n}{d}

\PY{n}{a}

\PY{n+nf}{match}\PY{p}{(}\PY{n}{a}\PY{p}{,}\PY{n+nf}{c}\PY{p}{(}\PY{l+m}{1}\PY{p}{,}\PY{l+m}{2}\PY{p}{)}\PY{p}{)}
\end{Verbatim}
\end{tcolorbox}

    \begin{enumerate*}
\item TRUE
\item TRUE
\item FALSE
\end{enumerate*}


    
    \begin{enumerate*}
\item 1
\item 2
\item 3
\end{enumerate*}


    
    \begin{enumerate*}
\item 1
\item 2
\item <NA>
\end{enumerate*}


    
    \begin{tcolorbox}[breakable, size=fbox, boxrule=1pt, pad at break*=1mm,colback=cellbackground, colframe=cellborder]
\prompt{In}{incolor}{116}{\boxspacing}
\begin{Verbatim}[commandchars=\\\{\}]
\PY{n}{a\PYZus{}list}

\PY{n+nf}{match}\PY{p}{(}\PY{n}{a\PYZus{}list}\PY{p}{,}\PY{n+nf}{c}\PY{p}{(}\PY{l+m}{4}\PY{p}{,}\PY{l+m}{5}\PY{p}{,}\PY{l+m}{6}\PY{p}{)}\PY{p}{)}

\PY{n+nf}{match}\PY{p}{(}\PY{n}{a\PYZus{}list}\PY{p}{,}\PY{n+nf}{list}\PY{p}{(}\PY{n+nf}{c}\PY{p}{(}\PY{l+m}{4}\PY{p}{,}\PY{l+m}{5}\PY{p}{,}\PY{l+m}{6}\PY{p}{)}\PY{p}{)}\PY{p}{)}

\PY{n+nf}{match}\PY{p}{(}\PY{l+s}{\PYZdq{}}\PY{l+s}{1\PYZdq{}}\PY{p}{,}\PY{n+nf}{c}\PY{p}{(}\PY{l+m}{3}\PY{p}{,}\PY{l+m}{5}\PY{p}{,}\PY{l+m}{8}\PY{p}{,}\PY{l+m}{1}\PY{p}{,}\PY{l+m}{99}\PY{p}{)}\PY{p}{)}      \PY{c+c1}{\PYZsh{} returns the position/index}
\end{Verbatim}
\end{tcolorbox}

    \begin{description}
\item[\$first] \begin{enumerate*}
\item 1
\item 2
\item 3
\end{enumerate*}

\item[\$second] \begin{enumerate*}
\item 4
\item 5
\item 6
\end{enumerate*}

\end{description}


    
    \begin{enumerate*}
\item <NA>
\item <NA>
\end{enumerate*}


    
    \begin{enumerate*}
\item <NA>
\item 1
\end{enumerate*}


    
    4

    
    \begin{tcolorbox}[breakable, size=fbox, boxrule=1pt, pad at break*=1mm,colback=cellbackground, colframe=cellborder]
\prompt{In}{incolor}{117}{\boxspacing}
\begin{Verbatim}[commandchars=\\\{\}]
\PY{n}{d}

\PY{n+nf}{grep}\PY{p}{(}\PY{l+s}{\PYZdq{}}\PY{l+s}{two\PYZdq{}}\PY{p}{,}\PY{n}{d}\PY{p}{)}   \PY{c+c1}{\PYZsh{} returns the position/index}
\end{Verbatim}
\end{tcolorbox}

    \begin{enumerate*}
\item 'one'
\item 'two'
\item 'three'
\item 'four'
\end{enumerate*}


    
    2

    
    \begin{tcolorbox}[breakable, size=fbox, boxrule=1pt, pad at break*=1mm,colback=cellbackground, colframe=cellborder]
\prompt{In}{incolor}{129}{\boxspacing}
\begin{Verbatim}[commandchars=\\\{\}]
\PY{c+c1}{\PYZsh{} B4. Statistics}

\PY{c+c1}{\PYZsh{} B4.1. Data}
\PY{n+nf}{set.seed}\PY{p}{(}\PY{l+m}{123456789}\PY{p}{)}

\PY{n}{a} \PY{o}{\PYZlt{}\PYZhy{}} \PY{n+nf}{c}\PY{p}{(}\PY{l+m}{1}\PY{o}{:}\PY{l+m}{1000}\PY{p}{)}
\PY{n}{b} \PY{o}{\PYZlt{}\PYZhy{}} \PY{n+nf}{c}\PY{p}{(}\PY{l+s}{\PYZdq{}}\PY{l+s}{one\PYZdq{}}\PY{p}{,}\PY{l+s}{\PYZdq{}}\PY{l+s}{two\PYZdq{}}\PY{p}{,}\PY{l+s}{\PYZdq{}}\PY{l+s}{NA\PYZdq{}}\PY{p}{,}\PY{l+m}{4}\PY{p}{,}\PY{l+m}{5}\PY{o}{:}\PY{l+m}{1000}\PY{p}{)}
\PY{n}{e} \PY{o}{\PYZlt{}\PYZhy{}} \PY{n+nf}{rnorm}\PY{p}{(}\PY{l+m}{1000}\PY{p}{)}
\PY{n}{c} \PY{o}{\PYZlt{}\PYZhy{}} \PY{l+m}{2} \PY{o}{\PYZhy{}} \PY{l+m}{3}\PY{o}{*}\PY{n}{a} \PY{o}{+} \PY{n}{e}
\PY{n}{x} \PY{o}{\PYZlt{}\PYZhy{}} \PY{n+nf}{as.data.frame}\PY{p}{(}\PY{n+nf}{cbind}\PY{p}{(}\PY{n}{a}\PY{p}{,}\PY{n}{b}\PY{p}{,}\PY{n}{c}\PY{p}{)}\PY{p}{)}   \PY{c+c1}{\PYZsh{} making it as a data frame consisting of 1000 rows and 3 columns (variables)}
                                   \PY{c+c1}{\PYZsh{} still has \PYZdq{}one\PYZdq{}, \PYZdq{}two\PYZdq{}, and missing variables }

\PY{n}{x} \PY{o}{\PYZlt{}\PYZhy{}} \PY{n+nf}{as.data.frame}\PY{p}{(}\PY{n+nf}{cbind}\PY{p}{(}\PY{n}{a}\PY{p}{,}\PY{n}{b}\PY{p}{,}\PY{n}{c}\PY{p}{)}\PY{p}{,} \PY{n}{stringsAsFactors} \PY{o}{=} \PY{k+kc}{FALSE}\PY{p}{)}
\end{Verbatim}
\end{tcolorbox}

    \begin{tcolorbox}[breakable, size=fbox, boxrule=1pt, pad at break*=1mm,colback=cellbackground, colframe=cellborder]
\prompt{In}{incolor}{134}{\boxspacing}
\begin{Verbatim}[commandchars=\\\{\}]
\PY{n}{x}\PY{o}{\PYZdl{}}\PY{n}{a} \PY{o}{\PYZlt{}\PYZhy{}} \PY{n+nf}{as.numeric}\PY{p}{(}\PY{n}{x}\PY{o}{\PYZdl{}}\PY{n}{a}\PY{p}{)}

\PY{c+c1}{\PYZsh{} x\PYZdl{}a       \PYZsh{} goes back to getting a(c:1000). without numeric, you get it in string}

\PY{n}{x}\PY{o}{\PYZdl{}}\PY{n}{c} \PY{o}{\PYZlt{}\PYZhy{}} \PY{n+nf}{as.numeric}\PY{p}{(}\PY{n}{x}\PY{o}{\PYZdl{}}\PY{n}{c}\PY{p}{)}

\PY{c+c1}{\PYZsh{} x\PYZdl{}c}

\PY{n+nf}{write.csv}\PY{p}{(}\PY{n}{x}\PY{p}{,} \PY{l+s}{\PYZdq{}}\PY{l+s}{x.csv\PYZdq{}}\PY{p}{)}
\end{Verbatim}
\end{tcolorbox}

    \begin{tcolorbox}[breakable, size=fbox, boxrule=1pt, pad at break*=1mm,colback=cellbackground, colframe=cellborder]
\prompt{In}{incolor}{137}{\boxspacing}
\begin{Verbatim}[commandchars=\\\{\}]
\PY{c+c1}{\PYZsh{} B4.2 Missing Values}

\PY{l+m}{2}\PY{o}{+}\PY{k+kc}{NA}

\PY{l+m}{2}\PY{o}{*}\PY{k+kc}{NA}

\PY{l+m}{2} \PY{o}{+} \PY{n+nf}{c}\PY{p}{(}\PY{l+m}{3}\PY{p}{,}\PY{l+m}{4}\PY{p}{,}\PY{l+m}{5}\PY{p}{,}\PY{l+m}{6}\PY{p}{,}\PY{l+m}{7}\PY{p}{,}\PY{k+kc}{NA}\PY{p}{)}
\end{Verbatim}
\end{tcolorbox}

    <NA>

    
    <NA>

    
    \begin{enumerate*}
\item 5
\item 6
\item 7
\item 8
\item 9
\item <NA>
\end{enumerate*}


    
    \begin{tcolorbox}[breakable, size=fbox, boxrule=1pt, pad at break*=1mm,colback=cellbackground, colframe=cellborder]
\prompt{In}{incolor}{150}{\boxspacing}
\begin{Verbatim}[commandchars=\\\{\}]
\PY{c+c1}{\PYZsh{} B4.3 Summary Statistics}

\PY{n+nf}{mean}\PY{p}{(}\PY{n}{x}\PY{o}{\PYZdl{}}\PY{n}{a}\PY{p}{)}

\PY{n+nf}{sd}\PY{p}{(}\PY{n}{x}\PY{o}{\PYZdl{}}\PY{n}{a}\PY{p}{)}

\PY{n+nf}{quantile}\PY{p}{(}\PY{n}{x}\PY{o}{\PYZdl{}}\PY{n}{a}\PY{p}{,} \PY{n+nf}{c}\PY{p}{(}\PY{l+m}{1}\PY{o}{:}\PY{l+m}{5}\PY{p}{)}\PY{o}{/}\PY{l+m}{5}\PY{p}{)}

\PY{c+c1}{\PYZsh{} rowMeans(x[,c(1,3)])}

\PY{n+nf}{colMeans}\PY{p}{(}\PY{n}{x}\PY{p}{[}\PY{p}{,}\PY{n+nf}{c}\PY{p}{(}\PY{l+m}{1}\PY{p}{,}\PY{l+m}{3}\PY{p}{)}\PY{p}{]}\PY{p}{)}   \PY{c+c1}{\PYZsh{} important! it takes a mean for all row in a column. }
                       \PY{c+c1}{\PYZsh{} in this case, we indicate we take a mean for column a and c}


\PY{n}{x}\PY{o}{\PYZdl{}}\PY{n}{d} \PY{o}{\PYZlt{}\PYZhy{}} \PY{k+kc}{NA}
\PY{n}{x}\PY{p}{[}\PY{l+m}{2}\PY{o}{:}\PY{l+m}{1000}\PY{p}{,}\PY{p}{]}\PY{o}{\PYZdl{}}\PY{n}{d} \PY{o}{\PYZlt{}\PYZhy{}} \PY{n+nf}{c}\PY{p}{(}\PY{l+m}{2}\PY{o}{:}\PY{l+m}{1000}\PY{o}{/}\PY{l+m}{10}\PY{p}{)}
\PY{n+nf}{mean}\PY{p}{(}\PY{n}{x}\PY{o}{\PYZdl{}}\PY{n}{d}\PY{p}{)}
\PY{n+nf}{mean}\PY{p}{(}\PY{n}{x}\PY{o}{\PYZdl{}}\PY{n}{d}\PY{p}{,} \PY{n}{na.rm}\PY{o}{=}\PY{k+kc}{TRUE}\PY{p}{)} \PY{c+c1}{\PYZsh{} mean when ignoring NA}
\end{Verbatim}
\end{tcolorbox}

    500.5

    
    288.819436095749

    
    \begin{description*}
\item[20\textbackslash{}\%] 200.8
\item[40\textbackslash{}\%] 400.6
\item[60\textbackslash{}\%] 600.4
\item[80\textbackslash{}\%] 800.2
\item[100\textbackslash{}\%] 1000
\end{description*}


    
    \begin{description*}
\item[a] 500.5
\item[c] -1499.48782712157
\end{description*}


    
    <NA>

    
    50.1

    
    \begin{tcolorbox}[breakable, size=fbox, boxrule=1pt, pad at break*=1mm,colback=cellbackground, colframe=cellborder]
\prompt{In}{incolor}{166}{\boxspacing}
\begin{Verbatim}[commandchars=\\\{\}]
\PY{c+c1}{\PYZsh{} B4.4 Regression}

\PY{n}{x} \PY{o}{\PYZlt{}\PYZhy{}} \PY{n+nf}{read.csv}\PY{p}{(}\PY{l+s}{\PYZdq{}}\PY{l+s}{x.csv\PYZdq{}}\PY{p}{,}\PY{n}{as.is}\PY{o}{=}\PY{k+kc}{TRUE}\PY{p}{)}
\PY{c+c1}{\PYZsh{} summary(x)}
\PY{n}{lm1} \PY{o}{\PYZlt{}\PYZhy{}} \PY{n+nf}{lm}\PY{p}{(}\PY{n}{c} \PY{o}{\PYZti{}} \PY{n}{a}\PY{p}{,} \PY{n}{data}\PY{o}{=}\PY{n}{x}\PY{p}{)}
\PY{c+c1}{\PYZsh{} summary(lm1)}

\PY{n+nf}{summary}\PY{p}{(}\PY{n}{lm1}\PY{p}{)}\PY{p}{[[}\PY{l+m}{4}\PY{p}{]]}   \PY{c+c1}{\PYZsh{} SUPER IMPORTANT!!! HEHE}

\PY{n}{lm1}\PY{o}{\PYZdl{}}\PY{n}{coefficients}    \PY{c+c1}{\PYZsh{} take just coefficients}

\PY{n}{glm1} \PY{o}{\PYZlt{}\PYZhy{}} \PY{n+nf}{glm}\PY{p}{(}\PY{n}{c} \PY{o}{\PYZgt{}} \PY{l+m}{\PYZhy{}1500} \PY{o}{\PYZti{}} \PY{n}{a}\PY{p}{,} \PY{n}{family} \PY{o}{=} \PY{n+nf}{binomial}\PY{p}{(}\PY{n}{link}\PY{o}{=}\PY{n}{probit}\PY{p}{)}\PY{p}{,} \PY{n}{data}\PY{o}{=}\PY{n}{x}\PY{p}{)}

\PY{n}{glm1}\PY{o}{\PYZdl{}}\PY{n}{coefficients}
\end{Verbatim}
\end{tcolorbox}

    \begin{tabular}{r|llll}
  & Estimate & Std. Error & t value & Pr(>\textbar{}t\textbar{})\\
\hline
	(Intercept) &  2.097396     & 0.0641312421  &     32.70475  & 4.934713e-160\\
	a & -3.000170     & 0.0001109953  & -27029.69905  &  0.000000e+00\\
\end{tabular}


    
    \begin{description*}
\item[(Intercept)] 2.0973962102833
\item[a] -3.00017027638733
\end{description*}


    
    \begin{Verbatim}[commandchars=\\\{\}]
Warning message:
"glm.fit: algorithm did not converge"Warning message:
"glm.fit: fitted probabilities numerically 0 or 1 occurred"
    \end{Verbatim}

    \begin{description*}
\item[(Intercept)] 3799.86765848464
\item[a] -7.59214317379549
\end{description*}


    
    \begin{tcolorbox}[breakable, size=fbox, boxrule=1pt, pad at break*=1mm,colback=cellbackground, colframe=cellborder]
\prompt{In}{incolor}{170}{\boxspacing}
\begin{Verbatim}[commandchars=\\\{\}]
\PY{c+c1}{\PYZsh{} B5. Control}

\PY{c+c1}{\PYZsh{} B5.1. Loops}

\PY{c+c1}{\PYZsh{} B5.2. Looping in R}

\PY{c+c1}{\PYZsh{} don\PYZsq{}t do it this way!!! WRONG EXAMPLE !!!}
\PY{n}{start\PYZus{}time} \PY{o}{\PYZlt{}\PYZhy{}} \PY{n+nf}{Sys.time}\PY{p}{(}\PY{p}{)}
\PY{n}{A} \PY{o}{\PYZlt{}\PYZhy{}} \PY{k+kc}{NULL}
\PY{n+nf}{for }\PY{p}{(}\PY{n}{i} \PY{n}{in} \PY{l+m}{1}\PY{o}{:}\PY{l+m}{10000}\PY{p}{)} \PY{p}{\PYZob{}}
    \PY{n}{A} \PY{o}{\PYZlt{}\PYZhy{}} \PY{n+nf}{rbind}\PY{p}{(}\PY{n}{A}\PY{p}{,}\PY{n+nf}{c}\PY{p}{(}\PY{n}{i}\PY{p}{,}\PY{n}{i}\PY{l+m}{+1}\PY{p}{,}\PY{n}{i}\PY{l+m}{+2}\PY{p}{)}\PY{p}{)}
\PY{p}{\PYZcb{}}
\PY{c+c1}{\PYZsh{} A}
\PY{n+nf}{Sys.time}\PY{p}{(}\PY{p}{)} \PY{o}{\PYZhy{}} \PY{n}{start\PYZus{}time}

\PY{c+c1}{\PYZsh{} A[400,]}
\end{Verbatim}
\end{tcolorbox}

    
    \begin{Verbatim}[commandchars=\\\{\}]
Time difference of 0.2747722 secs
    \end{Verbatim}

    
    \begin{enumerate*}
\item 400
\item 401
\item 402
\end{enumerate*}


    
    \begin{tcolorbox}[breakable, size=fbox, boxrule=1pt, pad at break*=1mm,colback=cellbackground, colframe=cellborder]
\prompt{In}{incolor}{175}{\boxspacing}
\begin{Verbatim}[commandchars=\\\{\}]
\PY{c+c1}{\PYZsh{} A faster Way}
\PY{n}{start\PYZus{}time} \PY{o}{\PYZlt{}\PYZhy{}} \PY{n+nf}{Sys.time}\PY{p}{(}\PY{p}{)}
\PY{n}{A} \PY{o}{\PYZlt{}\PYZhy{}} \PY{n+nf}{matrix}\PY{p}{(}\PY{k+kc}{NA}\PY{p}{,}\PY{l+m}{10000}\PY{p}{,}\PY{l+m}{3}\PY{p}{)}
\PY{n+nf}{for }\PY{p}{(}\PY{n}{i} \PY{n}{in} \PY{l+m}{1}\PY{o}{:}\PY{l+m}{10000}\PY{p}{)} \PY{p}{\PYZob{}}
    \PY{n}{A}\PY{p}{[}\PY{n}{i}\PY{p}{,}\PY{p}{]} \PY{o}{\PYZlt{}\PYZhy{}} \PY{n+nf}{c}\PY{p}{(}\PY{n}{i}\PY{p}{,}\PY{n}{i}\PY{l+m}{+1}\PY{p}{,}\PY{n}{i}\PY{l+m}{+2}\PY{p}{)}
\PY{p}{\PYZcb{}}
\PY{c+c1}{\PYZsh{} A}
\PY{n+nf}{Sys.time}\PY{p}{(}\PY{p}{)} \PY{o}{\PYZhy{}} \PY{n}{start\PYZus{}time}

\PY{n}{A}\PY{p}{[}\PY{l+m}{400}\PY{p}{,}\PY{p}{]}

\PY{n+nf}{sum}\PY{p}{(}\PY{n}{A}\PY{p}{)}
\end{Verbatim}
\end{tcolorbox}

    
    \begin{Verbatim}[commandchars=\\\{\}]
Time difference of 0.01392508 secs
    \end{Verbatim}

    
    \begin{enumerate*}
\item 400
\item 401
\item 402
\end{enumerate*}


    
    150045000

    
    \begin{tcolorbox}[breakable, size=fbox, boxrule=1pt, pad at break*=1mm,colback=cellbackground, colframe=cellborder]
\prompt{In}{incolor}{179}{\boxspacing}
\begin{Verbatim}[commandchars=\\\{\}]
\PY{c+c1}{\PYZsh{} An even faster way!!! (sometimes)}
\PY{n}{start\PYZus{}time} \PY{o}{\PYZlt{}\PYZhy{}} \PY{n+nf}{Sys.time}\PY{p}{(}\PY{p}{)}
\PY{n}{A} \PY{o}{\PYZlt{}\PYZhy{}} \PY{n+nf}{t}\PY{p}{(}\PY{n+nf}{matrix}\PY{p}{(}\PY{n+nf}{sapply}\PY{p}{(}\PY{l+m}{1}\PY{o}{:}\PY{l+m}{10000}\PY{p}{,}
                    \PY{n+nf}{function}\PY{p}{(}\PY{n}{x}\PY{p}{)} \PY{n+nf}{c}\PY{p}{(}\PY{n}{x}\PY{p}{,}\PY{n}{x}\PY{l+m}{+1}\PY{p}{,}\PY{n}{x}\PY{l+m}{+2}\PY{p}{)}\PY{p}{)}\PY{p}{,} \PY{n}{nrow}\PY{o}{=}\PY{l+m}{3}\PY{p}{)}\PY{p}{)}

\PY{n+nf}{Sys.time}\PY{p}{(}\PY{p}{)} \PY{o}{\PYZhy{}} \PY{n}{start\PYZus{}time}
\PY{c+c1}{\PYZsh{} sapply is similar of doing \PYZdq{}for loop\PYZdq{} for matrix}

\PY{n}{A}\PY{p}{[}\PY{l+m}{400}\PY{p}{,}\PY{p}{]}
\PY{n+nf}{sum}\PY{p}{(}\PY{n}{A}\PY{p}{)}
\end{Verbatim}
\end{tcolorbox}

    
    \begin{Verbatim}[commandchars=\\\{\}]
Time difference of 0.0149281 secs
    \end{Verbatim}

    
    \begin{enumerate*}
\item 400
\item 401
\item 402
\end{enumerate*}


    
    150045000

    
    \begin{tcolorbox}[breakable, size=fbox, boxrule=1pt, pad at break*=1mm,colback=cellbackground, colframe=cellborder]
\prompt{In}{incolor}{183}{\boxspacing}
\begin{Verbatim}[commandchars=\\\{\}]
\PY{c+c1}{\PYZsh{} B5.3 IF ELSE}

\PY{n}{a} \PY{o}{\PYZlt{}\PYZhy{}} \PY{n+nf}{c}\PY{p}{(}\PY{l+m}{1}\PY{p}{,}\PY{l+m}{2}\PY{p}{,}\PY{l+m}{3}\PY{p}{,}\PY{l+m}{4}\PY{p}{,}\PY{l+m}{5}\PY{p}{)}
\PY{n}{b} \PY{o}{\PYZlt{}\PYZhy{}} \PY{n+nf}{ifelse}\PY{p}{(}\PY{n}{a}\PY{o}{==}\PY{l+m}{3}\PY{p}{,}\PY{l+m}{82}\PY{p}{,}\PY{n}{a}\PY{p}{)}
\PY{n}{a}
\PY{n}{b}

\PY{n}{A} \PY{o}{\PYZlt{}\PYZhy{}} \PY{l+s}{\PYZdq{}}\PY{l+s}{Chris\PYZdq{}}
\PY{n+nf}{if }\PY{p}{(}\PY{n}{A}\PY{o}{==}\PY{l+s}{\PYZdq{}}\PY{l+s}{Chris\PYZdq{}}\PY{p}{)} \PY{p}{\PYZob{}}
    \PY{n+nf}{print}\PY{p}{(}\PY{l+s}{\PYZdq{}}\PY{l+s}{Hey Chris\PYZdq{}}\PY{p}{)}
\PY{p}{\PYZcb{}} \PY{n}{else} \PY{p}{\PYZob{}}
    \PY{n+nf}{print}\PY{p}{(}\PY{n+nf}{paste}\PY{p}{(}\PY{l+s}{\PYZdq{}}\PY{l+s}{Hey\PYZdq{}}\PY{p}{,}\PY{n}{A}\PY{p}{)}\PY{p}{)}
\PY{p}{\PYZcb{}}
\end{Verbatim}
\end{tcolorbox}

    \begin{enumerate*}
\item 1
\item 2
\item 3
\item 4
\item 5
\end{enumerate*}


    
    \begin{enumerate*}
\item 1
\item 2
\item 82
\item 4
\item 5
\end{enumerate*}


    
    \begin{Verbatim}[commandchars=\\\{\}]
[1] "Hey Chris"
    \end{Verbatim}

    \begin{tcolorbox}[breakable, size=fbox, boxrule=1pt, pad at break*=1mm,colback=cellbackground, colframe=cellborder]
\prompt{In}{incolor}{191}{\boxspacing}
\begin{Verbatim}[commandchars=\\\{\}]
\PY{c+c1}{\PYZsh{} B.6 Optimization}

\PY{c+c1}{\PYZsh{} B6.1 Function}
\PY{n}{y} \PY{o}{\PYZlt{}\PYZhy{}} \PY{n}{x}\PY{p}{[}\PY{p}{,}\PY{n+nf}{c}\PY{p}{(}\PY{l+m}{2}\PY{p}{,}\PY{l+m}{4}\PY{p}{)}\PY{p}{]}    \PY{c+c1}{\PYZsh{} from dataset x.csv, (1000x2) taking only a and c}

\PY{n+nf}{apply}\PY{p}{(}\PY{n}{y}\PY{p}{,}\PY{l+m}{2}\PY{p}{,}\PY{n}{mean}\PY{p}{)}

\PY{n+nf}{colMeans}\PY{p}{(}\PY{n}{y}\PY{p}{)}       \PY{c+c1}{\PYZsh{} in this case, colMeans(y) is equivalent to apply(y,2,mean)}

\PY{n}{b} \PY{o}{\PYZlt{}\PYZhy{}} \PY{n+nf}{c}\PY{p}{(}\PY{l+m}{1}\PY{o}{:}\PY{n+nf}{dim}\PY{p}{(}\PY{n}{y}\PY{p}{)}\PY{p}{[}\PY{l+m}{1}\PY{p}{]}\PY{p}{)}

\PY{n+nf}{summary}\PY{p}{(}\PY{n+nf}{sapply}\PY{p}{(}\PY{n}{b}\PY{p}{,}\PY{n+nf}{function}\PY{p}{(}\PY{n}{x}\PY{p}{)} \PY{n+nf}{sum}\PY{p}{(}\PY{n}{y}\PY{p}{[}\PY{n}{x}\PY{p}{,}\PY{p}{]}\PY{p}{)}\PY{p}{)}\PY{p}{,}\PY{n}{digits}\PY{o}{=}\PY{l+m}{2}\PY{p}{)}
               
\PY{n+nf}{summary}\PY{p}{(}\PY{n+nf}{rowSums}\PY{p}{(}\PY{n}{y}\PY{p}{)}\PY{p}{,} \PY{n}{digits} \PY{o}{=} \PY{l+m}{2}\PY{p}{)}
\end{Verbatim}
\end{tcolorbox}

    \begin{description*}
\item[a] 500.5
\item[c] -1499.48782712157
\end{description*}


    
    \begin{description*}
\item[a] 500.5
\item[c] -1499.48782712157
\end{description*}


    
    
    \begin{Verbatim}[commandchars=\\\{\}]
   Min. 1st Qu.  Median    Mean 3rd Qu.    Max. 
-2000.0 -1500.0 -1000.0 -1000.0  -500.0     0.5 
    \end{Verbatim}

    
    
    \begin{Verbatim}[commandchars=\\\{\}]
   Min. 1st Qu.  Median    Mean 3rd Qu.    Max. 
-2000.0 -1500.0 -1000.0 -1000.0  -500.0     0.5 
    \end{Verbatim}

    
    \begin{tcolorbox}[breakable, size=fbox, boxrule=1pt, pad at break*=1mm,colback=cellbackground, colframe=cellborder]
\prompt{In}{incolor}{197}{\boxspacing}
\begin{Verbatim}[commandchars=\\\{\}]
\PY{n}{my\PYZus{}mean} \PY{o}{\PYZlt{}\PYZhy{}} \PY{n+nf}{function}\PY{p}{(}\PY{n}{x}\PY{p}{)} \PY{p}{\PYZob{}}
    \PY{n+nf}{if }\PY{p}{(}\PY{n+nf}{is.numeric}\PY{p}{(}\PY{n}{x}\PY{p}{)}\PY{p}{)} \PY{p}{\PYZob{}}
        \PY{n+nf}{return}\PY{p}{(}\PY{n+nf}{mean}\PY{p}{(}\PY{n}{x}\PY{p}{,} \PY{n}{na.rm}\PY{o}{=}\PY{k+kc}{TRUE}\PY{p}{)}\PY{p}{)}
    \PY{p}{\PYZcb{}} 
    \PY{n}{else} \PY{n+nf}{return}\PY{p}{(}\PY{l+s}{\PYZdq{}}\PY{l+s}{Not Numeric!\PYZdq{}}\PY{p}{)}
\PY{p}{\PYZcb{}}
\PY{c+c1}{\PYZsh{} x\PYZdl{}b}
\PY{n+nf}{my\PYZus{}mean}\PY{p}{(}\PY{n}{x}\PY{o}{\PYZdl{}}\PY{n}{b}\PY{p}{)}

\PY{n+nf}{my\PYZus{}mean}\PY{p}{(}\PY{n}{x}\PY{o}{\PYZdl{}}\PY{n}{a}\PY{p}{)}

\PY{n+nf}{my\PYZus{}mean}\PY{p}{(}\PY{n}{x}\PY{o}{\PYZdl{}}\PY{n}{c}\PY{p}{)}
\end{Verbatim}
\end{tcolorbox}

    'Not Numeric!'

    
    500.5

    
    -1499.48782712157

    
    \begin{tcolorbox}[breakable, size=fbox, boxrule=1pt, pad at break*=1mm,colback=cellbackground, colframe=cellborder]
\prompt{In}{incolor}{203}{\boxspacing}
\begin{Verbatim}[commandchars=\\\{\}]
\PY{n}{lm\PYZus{}iv} \PY{o}{\PYZlt{}\PYZhy{}} \PY{n+nf}{function}\PY{p}{(}\PY{n}{y\PYZus{}in}\PY{p}{,} \PY{n}{X\PYZus{}in}\PY{p}{,} \PY{n}{Z\PYZus{}in} \PY{o}{=} \PY{n}{X\PYZus{}in}\PY{p}{,} \PY{n}{Reps} \PY{o}{=} \PY{l+m}{100}\PY{p}{,} \PY{n}{min\PYZus{}in} \PY{o}{=} \PY{l+m}{0.05}\PY{p}{,} \PY{n}{max\PYZus{}in} \PY{o}{=} \PY{l+m}{0.95}\PY{p}{)} \PY{p}{\PYZob{}}
    \PY{c+c1}{\PYZsh{} takes in the y variable, x explanatory varibales}
    \PY{c+c1}{\PYZsh{} and the z variables if available.}
    \PY{c+c1}{\PYZsh{} defaults: Z\PYZus{}in = X\PYZus{}in,}
    \PY{c+c1}{\PYZsh{} Reps = 100, min\PYZus{}in = 0.05, max\PYZus{}in = 0.95}
    
    \PY{c+c1}{\PYZsh{} Set up}
    \PY{n+nf}{set.seed}\PY{p}{(}\PY{l+m}{123456789}\PY{p}{)}
    \PY{n}{index\PYZus{}na} \PY{o}{\PYZlt{}\PYZhy{}} \PY{n+nf}{is.na}\PY{p}{(}\PY{n+nf}{rowSums}\PY{p}{(}\PY{n+nf}{cbind}\PY{p}{(}\PY{n}{y\PYZus{}in}\PY{p}{,}\PY{n}{X\PYZus{}in}\PY{p}{,}\PY{n}{Z\PYZus{}in}\PY{p}{)}\PY{p}{)}\PY{p}{)}  \PY{c+c1}{\PYZsh{} cbind y, X, Z. then do rowSums. then check NA. if yes, return ==1}
    \PY{n}{yt} \PY{o}{\PYZlt{}\PYZhy{}} \PY{n+nf}{as.matrix}\PY{p}{(}\PY{n}{y\PYZus{}in}\PY{p}{[}\PY{n}{index\PYZus{}na}\PY{o}{==}\PY{l+m}{0}\PY{p}{]}\PY{p}{)}
    \PY{n}{Xt} \PY{o}{\PYZlt{}\PYZhy{}} \PY{n+nf}{as.matrix}\PY{p}{(}\PY{n+nf}{cbind}\PY{p}{(}\PY{l+m}{1}\PY{p}{,}\PY{n}{X\PYZus{}in}\PY{p}{)}\PY{p}{)}
    \PY{n}{Xt} \PY{o}{\PYZlt{}\PYZhy{}} \PY{n}{Xt}\PY{p}{[}\PY{n}{index\PYZus{}na}\PY{o}{==}\PY{l+m}{0}\PY{p}{]}
    \PY{n}{Zt} \PY{o}{\PYZlt{}\PYZhy{}} \PY{n+nf}{as.matrix}\PY{p}{(}\PY{n+nf}{cbind}\PY{p}{(}\PY{l+m}{1}\PY{p}{,}\PY{n}{Z\PYZus{}in}\PY{p}{)}\PY{p}{)}
    \PY{n}{Zt} \PY{o}{\PYZlt{}\PYZhy{}} \PY{n}{Zt}\PY{p}{[}\PY{n}{index\PYZus{}na}\PY{o}{==}\PY{l+m}{0}\PY{p}{]}
    \PY{n}{N\PYZus{}temp} \PY{o}{\PYZlt{}\PYZhy{}} \PY{n+nf}{length}\PY{p}{(}\PY{n}{yt}\PY{p}{)}
    
    \PY{c+c1}{\PYZsh{} turns the inputs into matrices}
    \PY{c+c1}{\PYZsh{} removes observations with any missing values}
    \PY{c+c1}{\PYZsh{} add column of 1s to X and Z}
    
    \PY{c+c1}{\PYZsh{} Bootstrap}
    \PY{n}{r} \PY{o}{\PYZlt{}\PYZhy{}} \PY{n+nf}{c}\PY{p}{(}\PY{l+m}{1}\PY{o}{:}\PY{n}{Reps}\PY{p}{)}
    \PY{n}{bs\PYZus{}temp} \PY{o}{\PYZlt{}\PYZhy{}} \PY{n+nf}{sapply}\PY{p}{(}\PY{n}{r}\PY{p}{,} \PY{n+nf}{function}\PY{p}{(}\PY{n}{x}\PY{p}{)} \PY{p}{\PYZob{}}
        \PY{n}{ibs} \PY{o}{\PYZlt{}\PYZhy{}} \PY{n+nf}{round}\PY{p}{(}\PY{n+nf}{runif}\PY{p}{(}\PY{n}{N\PYZus{}temp}\PY{p}{,} \PY{n}{min} \PY{o}{=} \PY{l+m}{1}\PY{p}{,} \PY{n}{max} \PY{o}{=} \PY{n}{N\PYZus{}temp}\PY{p}{)}\PY{p}{)}
        \PY{n+nf}{solve}\PY{p}{(} \PY{n+nf}{t}\PY{p}{(}\PY{n}{Zt}\PY{p}{[}\PY{n}{ibs}\PY{p}{,}\PY{p}{]}\PY{p}{)}\PY{o}{\PYZpc{}*\PYZpc{}}\PY{n}{Xt}\PY{p}{[}\PY{n}{ibs}\PY{p}{,}\PY{p}{]} \PY{p}{)}\PY{o}{\PYZpc{}*\PYZpc{}}\PY{n+nf}{t}\PY{p}{(}\PY{n}{Zt}\PY{p}{[}\PY{n}{ibs}\PY{p}{,}\PY{p}{]}\PY{p}{)}\PY{o}{\PYZpc{}*\PYZpc{}}\PY{n}{yt}\PY{p}{[}\PY{n}{ibs}\PY{p}{]}
    \PY{p}{\PYZcb{}} \PY{p}{)}
    
    \PY{c+c1}{\PYZsh{} Present Results}
    \PY{n}{res\PYZus{}temp} \PY{o}{\PYZlt{}\PYZhy{}} \PY{n+nf}{matrix}\PY{p}{(}\PY{k+kc}{NA}\PY{p}{,}\PY{n+nf}{dim}\PY{p}{(}\PY{n}{Xt}\PY{p}{)}\PY{p}{)}
    \PY{n}{res\PYZus{}temp}\PY{p}{[}\PY{p}{,}\PY{l+m}{1}\PY{p}{]} \PY{o}{\PYZlt{}\PYZhy{}} \PY{n+nf}{rowMeans}\PY{p}{(}\PY{n}{bs\PYZus{}temp}\PY{p}{)}
    \PY{n+nf}{for }\PY{p}{(}\PY{n}{j} \PY{n}{in} \PY{l+m}{1}\PY{o}{:}\PY{n+nf}{dim}\PY{p}{(}\PY{n}{Xt}\PY{p}{)}\PY{p}{[}\PY{l+m}{2}\PY{p}{]}\PY{p}{)} \PY{p}{\PYZob{}}
        \PY{n}{res\PYZus{}temp}\PY{p}{[}\PY{n}{j}\PY{p}{,}\PY{l+m}{2}\PY{p}{]} \PY{o}{\PYZlt{}\PYZhy{}} \PY{n+nf}{sd}\PY{p}{(}\PY{n}{bs\PYZus{}temp}\PY{p}{[}\PY{n}{j}\PY{p}{,}\PY{p}{]}\PY{p}{)}
        \PY{n}{res\PYZus{}temp}\PY{p}{[}\PY{n}{j}\PY{p}{,}\PY{l+m}{3}\PY{p}{]} \PY{o}{\PYZlt{}\PYZhy{}} \PY{n+nf}{quantile}\PY{p}{(}\PY{n}{bs\PYZus{}temp}\PY{p}{[}\PY{n}{j}\PY{p}{,}\PY{p}{]}\PY{p}{,}\PY{n}{min\PYZus{}in}\PY{p}{)}
        \PY{n}{res\PYZus{}temp}\PY{p}{[}\PY{n}{j}\PY{p}{,}\PY{l+m}{4}\PY{p}{]} \PY{o}{\PYZlt{}\PYZhy{}} \PY{n+nf}{quantile}\PY{p}{(}\PY{n}{bs\PYZus{}temp}\PY{p}{[}\PY{n}{j}\PY{p}{,}\PY{p}{]}\PY{p}{,}\PY{n}{max\PYZus{}in}\PY{p}{)}
    \PY{p}{\PYZcb{}}
    
    \PY{n+nf}{colnames}\PY{p}{(}\PY{n}{res\PYZus{}temp}\PY{p}{)} \PY{o}{\PYZlt{}\PYZhy{}} \PY{n+nf}{c}\PY{p}{(}\PY{l+s}{\PYZdq{}}\PY{l+s}{coef\PYZdq{}}\PY{p}{,}\PY{l+s}{\PYZdq{}}\PY{l+s}{sd\PYZdq{}}\PY{p}{,}\PY{n+nf}{as.character}\PY{p}{(}\PY{n}{min\PYZus{}in}\PY{p}{)}\PY{p}{,}\PY{n+nf}{as.character}\PY{p}{(}\PY{n}{max\PYZus{}in}\PY{p}{)}\PY{p}{)}
    \PY{n+nf}{return}\PY{p}{(}\PY{n}{res\PYZus{}temp}\PY{p}{)}
\PY{p}{\PYZcb{}}

\PY{c+c1}{\PYZsh{} lmiv1  \PYZlt{}\PYZhy{} lm\PYZus{}iv(x\PYZdl{}c, x\PYZdl{}a)}
\PY{c+c1}{\PYZsh{} lmiv1}
\PY{c+c1}{\PYZsh{} IT DOES NOT WORK!!!! NOT YET MITIGATED}
\end{Verbatim}
\end{tcolorbox}

    \begin{Verbatim}[commandchars=\\\{\}, frame=single, framerule=2mm, rulecolor=\color{outerrorbackground}]
Error in Zt[ibs, ]: incorrect number of dimensions
Traceback:

1. lm\_iv(x\$c, x\$a)
2. sapply(r, function(x) \{
 .     ibs <- round(runif(N\_temp, min = 1, max = N\_temp))
 .     solve(t(Zt[ibs, ]) \%*\% Xt[ibs, ]) \%*\% t(Zt[ibs, ]) \%*\% yt[ibs]
 . \})   \# at line 23-26 of file <text>
3. lapply(X = X, FUN = FUN, {\ldots})
4. FUN(X[[i]], {\ldots})
5. solve(t(Zt[ibs, ]) \%*\% Xt[ibs, ])   \# at line 25 of file <text>
6. t(Zt[ibs, ])   \# at line 25 of file <text>
    \end{Verbatim}

    \begin{tcolorbox}[breakable, size=fbox, boxrule=1pt, pad at break*=1mm,colback=cellbackground, colframe=cellborder]
\prompt{In}{incolor}{208}{\boxspacing}
\begin{Verbatim}[commandchars=\\\{\}]
\PY{c+c1}{\PYZsh{} B6.2 optim()}

\PY{n}{f\PYZus{}ols} \PY{o}{\PYZlt{}\PYZhy{}} \PY{n+nf}{function}\PY{p}{(}\PY{n}{beta}\PY{p}{,} \PY{n}{y\PYZus{}in}\PY{p}{,} \PY{n}{X\PYZus{}in}\PY{p}{)} \PY{p}{\PYZob{}}
    \PY{n}{X\PYZus{}in} \PY{o}{\PYZlt{}\PYZhy{}} \PY{n+nf}{as.matrix}\PY{p}{(}\PY{n+nf}{cbind}\PY{p}{(}\PY{l+m}{1}\PY{p}{,}\PY{n}{X\PYZus{}in}\PY{p}{)}\PY{p}{)}
    \PY{n+nf}{if }\PY{p}{(}\PY{n+nf}{length}\PY{p}{(}\PY{n}{beta}\PY{p}{)}\PY{o}{==}\PY{n+nf}{dim}\PY{p}{(}\PY{n}{X\PYZus{}in}\PY{p}{)}\PY{p}{[}\PY{l+m}{2}\PY{p}{]}\PY{p}{)} \PY{p}{\PYZob{}}
        \PY{n+nf}{return}\PY{p}{(}\PY{n+nf}{mean}\PY{p}{(}\PY{p}{(}\PY{n}{y\PYZus{}in} \PY{o}{\PYZhy{}} \PY{n}{X\PYZus{}in}\PY{o}{\PYZpc{}*\PYZpc{}}\PY{n}{beta}\PY{p}{)}\PY{o}{\PYZca{}}\PY{l+m}{2}\PY{p}{,} \PY{n}{na.rm} \PY{o}{=} \PY{k+kc}{TRUE}\PY{p}{)}\PY{p}{)}
    \PY{p}{\PYZcb{}}
    \PY{n}{else} \PY{p}{\PYZob{}}
        \PY{n+nf}{return}\PY{p}{(}\PY{l+s}{\PYZdq{}}\PY{l+s}{The number of parameters does not match.\PYZdq{}}\PY{p}{)}
    \PY{p}{\PYZcb{}}
\PY{p}{\PYZcb{}}

\PY{n}{lm\PYZus{}ols} \PY{o}{\PYZlt{}\PYZhy{}} \PY{n+nf}{optim}\PY{p}{(}\PY{n}{par}\PY{o}{=}\PY{n+nf}{c}\PY{p}{(}\PY{l+m}{2}\PY{p}{,}\PY{l+m}{\PYZhy{}3}\PY{p}{)}\PY{p}{,}\PY{n}{fn}\PY{o}{=}\PY{n}{f\PYZus{}ols}\PY{p}{,}\PY{n}{y\PYZus{}in}\PY{o}{=}\PY{n}{x}\PY{o}{\PYZdl{}}\PY{n}{c}\PY{p}{,}\PY{n}{X\PYZus{}in}\PY{o}{=}\PY{n}{x}\PY{o}{\PYZdl{}}\PY{n}{a}\PY{p}{)}
\PY{n}{lm\PYZus{}ols}

\PY{n+nf}{lm}\PY{p}{(}\PY{n}{x}\PY{o}{\PYZdl{}}\PY{n}{c} \PY{o}{\PYZti{}} \PY{n}{x}\PY{o}{\PYZdl{}}\PY{n}{a}\PY{p}{)}

\PY{n}{lm\PYZus{}ols1} \PY{o}{\PYZlt{}\PYZhy{}} \PY{n+nf}{optim}\PY{p}{(}\PY{n}{par}\PY{o}{=}\PY{n+nf}{c}\PY{p}{(}\PY{l+m}{1}\PY{p}{,}\PY{l+m}{1}\PY{p}{)}\PY{p}{,}\PY{n}{fn}\PY{o}{=}\PY{n}{f\PYZus{}ols}\PY{p}{,}\PY{n}{y\PYZus{}in}\PY{o}{=}\PY{n}{x}\PY{o}{\PYZdl{}}\PY{n}{c}\PY{p}{,}\PY{n}{X\PYZus{}in}\PY{o}{=}\PY{n}{x}\PY{o}{\PYZdl{}}\PY{n}{a}\PY{p}{)}
\PY{n}{lm\PYZus{}ols1}
\end{Verbatim}
\end{tcolorbox}

    \begin{description}
\item[\$par] \begin{enumerate*}
\item 2.09728736536897
\item -3.00017009107694
\end{enumerate*}

\item[\$value] 1.02460919703226
\item[\$counts] \begin{description*}
\item[function] 67
\item[gradient] <NA>
\end{description*}

\item[\$convergence] 0
\item[\$message] NULL
\end{description}


    
    
    \begin{Verbatim}[commandchars=\\\{\}]

Call:
lm(formula = x\$c \textasciitilde{} x\$a)

Coefficients:
(Intercept)          x\$a  
      2.097       -3.000  

    \end{Verbatim}

    
    \begin{description}
\item[\$par] \begin{enumerate*}
\item 1.67590292801324
\item -2.99936153875897
\end{enumerate*}

\item[\$value] 1.07939341375334
\item[\$counts] \begin{description*}
\item[function] 49
\item[gradient] <NA>
\end{description*}

\item[\$convergence] 0
\item[\$message] NULL
\end{description}


    
    \begin{tcolorbox}[breakable, size=fbox, boxrule=1pt, pad at break*=1mm,colback=cellbackground, colframe=cellborder]
\prompt{In}{incolor}{ }{\boxspacing}
\begin{Verbatim}[commandchars=\\\{\}]

\end{Verbatim}
\end{tcolorbox}


    % Add a bibliography block to the postdoc
    
    
    
\end{document}
